\documentclass[11pt]{article}

\usepackage{amsmath,amssymb,amsfonts}
\usepackage{graphicx}
\usepackage[dvipsnames]{xcolor}
\usepackage{}


\newcommand{\norm}[1]{\left\lVert#1\right\rVert}
\newcommand{\abs}[1]{\left|#1\right|}

\setlength{\topmargin}{-.5in} \setlength{\textheight}{9.25in}
\setlength{\oddsidemargin}{0in} \setlength{\textwidth}{6.8in}


\begin{document}

\Large


\noindent{\bf Kernfragen - Reihen\hfill WS21-22 Analysis I.}


\medskip\hrule
\begin{enumerate}
    \item Wann heißt eine Reihe konvergent, wann absolut konvergent?
    \item  Für welche komplexen $q$ existiert $\sum_{n=0}^\infty {n^n}$? Welchen Wert hat die Summe?
    \item  Warum divergiert die harmonische Reihe?
    \item  Wann konvergiert eine Reihe positiver Summanden?
    \item  Wie lauten Cauchy-,Majoranten-,Verdichtungs- und Leibniz-Kriteriumfür die Konvergenz unendlicher Reihen?
    \item  Wie lauten Wurzel- und Quotientenkriterium für die Konvergenz unendlicher Reihen?
    \item  Bei welchen der folgenden Reihen gibt das Quotientenkriterium Aufschluss über Konvergenz oder Divergenz?
    \begin{center}
        $\sum_{n=0}^\infty \frac{n!}{n^n}$, $\sum_{n=0}^\infty \frac{1}{n^2}$, $\sum_{n=0}^\infty \frac{1}{(3+(-1)^n)^n}$
    \end{center}
    \item  Wie lautet der kleine Umordnungssatz absolut konvergenter Reihen?
    \item  Wie lautet der große Umordnungssatz absolut konvergenter Reihen?
    \item  Welche der folgenden Reihen konvergieren, welche konvergieren absolut?
    \begin{center}
        $\sum_{n=0}^\infty {\frac{1}{n}}$, $\sum_{n=0}^\infty \frac{(-1)^n}{n}$, $\sum_{n=0}^\infty \frac{(-1)^n}{n^2}$, $\sum_{n=0}^\infty \frac{x^n}{n!} \in \mathbb{C}$
    \end{center}
    \item  Für welche reellen/komplexen $s$ konvergiert die Reihe $\sum_{n=0}^\infty n^{-s}$ der Riemannschen
    $\zeta$-Funktion?
    \item  Was ist eine Potenzreihe? Was ist ihr Konvergenzradius? Wie berechnet er sich?
    \item  Wann ist das Produkt zweier Potenzreihen wieder eine Potenzreihe? Wie lautet sie? Wie hängen die Konvergenzradien der Potenzreihen und ihres Produktes zusammen?
    \item  Wie lauten die Dastellungen von $\exp(x), \sin(x), \cos(x), \sinh(x), \cosh(x)$ als Potenzreihen?
    \item  Wie hängen $e^z, \sin(z), \cos(z), \sinh(z), \cosh(z)$ im Komplexen zusammen?

\end{enumerate}

\end{document} 
