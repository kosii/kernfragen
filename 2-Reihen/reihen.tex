\documentclass[11pt]{article}

\usepackage{amsmath,amssymb,amsfonts}
\usepackage{graphicx}
\usepackage[dvipsnames]{xcolor}
\usepackage{}


\newcommand{\norm}[1]{\left\lVert#1\right\rVert}
\newcommand{\abs}[1]{\left|#1\right|}
\newcommand{\sumn}[4]{\sum_{#1=#2}^{#3}{#4}}
\newcommand{\RR}[0]{\mathbb{R}}
\newcommand{\CC}[0]{\mathbb{C}}
\newcommand{\QQ}[0]{\mathbb{Q}}
\newcommand{\ZZ}[0]{\mathbb{Z}}
\newcommand{\NN}[0]{\mathbb{N}}
\newcommand{\KK}[0]{\mathbb{K}}

\setlength{\topmargin}{-.5in} \setlength{\textheight}{9.25in}
\setlength{\oddsidemargin}{0in} \setlength{\textwidth}{6.8in}


\begin{document}

\Large


\noindent{\bf Kernfragen - Reihen\hfill WS21-22 Analysis I.}


\medskip\hrule
\begin{enumerate}
    \item Wann heißt eine Reihe konvergent, wann absolut konvergent?
    
    \textbf{Answer:} The series $\sumn{n}{0}{\infty}{a_n}$ converges when the sequence of partial sums $s_n = \sumn{k}{0}{n}{a_k}$ converges. The series $\sumn{n}{0}{\infty}{a_n}$ converges absolutely, when the series $\sumn{n}{0}{\infty}{\abs{a_n}}$ converges.

    \item  Für welche komplexen $q$ existiert $\sum_{n=0}^\infty {q^n}$? Welchen Wert hat die Summe?
    
    \textbf{Answer:} It exists for $\abs{q} < 1$, and $\sum_{n=0}^\infty {q^n} = \frac{1}{1-q}$. 
    
    $$s_n = s_{n-1} + 1, s_{n-1} = s_n - q^n \Rightarrow s_n = q(s_n - q^n)+1$$
    thus $s_n = \frac{1-q^{n+1}}{1-q}$. $s_n$ converges exactly when $\abs{q} < q$.

    \item  Warum divergiert die harmonische Reihe?
    
    \textbf{Answer:} For a similar argument that is used in the proof of the Verdichtungs-Kriterium: $\sumn{k}{1}{2^n}{\frac{1}{k}} > \sumn{k}{0}{n}{2^n \frac{1}{2^n}} = n$ (note: indexes might be off-by-one, but this is the main idea).

    \item  Wann konvergiert eine Reihe positiver Summanden?
    
    \textbf{Answer:} When sequence of partial sums is bounded.

    \item  Wie lauten Cauchy-, Majoranten-, Verdichtungs- und Leibniz-Kriterium für die Konvergenz unendlicher Reihen?
    
    \textbf{TODO: only from the top of my head, compare it against the lecture notes}
    \textbf{Answer:} \begin{itemize}
        \item Cauchy-criterium: $\sumn{n}{0}{\infty}{a_n}$ converges exactly if $\forall \epsilon > 0\colon  \exists N \in \mathbb{N}\colon \forall m,n > N\colon \abs{\sumn{k}{0}{n}{a_k} - \sumn{k}{m}{m}{a_k}} = \abs{\sumn{k}{m+1}{n}{a_k}} < \epsilon$
        \item Majorant: consider two nonnegative series $\sumn{n}{0}{\infty}{a_n}$ and $\sumn{n}{0}{\infty}{b_n}$. If $0 \ge a_n \le b_n$ for almost all $n\in \mathbb{N}$ and $\sumn{n}{0}{\infty}{b_n}$ converges, then so is $\sumn{n}{0}{\infty}{a_n}$.
        \item Verdichtungs: Consider $(a_n) \ge 0$ monoton decreasing terms. Then $\sumn{n}{0}{\infty}{a_n}$ converges exactly when $\sumn{n}{0}{\infty}{2^n a_{2^n}}$
        \item Leibniz: Consider $(a_n) \ge 0$ monoton decreasing. Then $\sumn{n}{0}{\infty}{(-1)^n a_n}$ converges.
    \end{itemize}

    \textbf{TODO: only from the top of my head, compare it against the lecture notes}
    \item  Wie lauten Wurzel- und Quotientenkriterium für die Konvergenz unendlicher Reihen?
    
    \textbf{Answer:}
    \begin{itemize}
        \item Root-test: if $\limsup_{n\to\infty}\abs{a_n}^{\frac{1}{n}} < 1$ then $\sumn{n}{0}{\infty}{a_n}$ converges.
        \item Ratio-test: if $a_n = 0$ for at most finitely many $n\in\mathbb{N}$ and $\limsup_{n\to\infty}\abs{\frac{a_{n+1}}{a_n}} < 1$ then $\sumn{n}{0}{\infty}{a_n}$ converges.
    \end{itemize}

    \item  Bei welchen der folgenden Reihen gibt das Quotientenkriterium Aufschluss über Konvergenz oder Divergenz?
    \begin{center}
        $\sum_{n=0}^\infty \frac{n!}{n^n}$, $\sum_{n=0}^\infty \frac{1}{n^2}$, $\sum_{n=0}^\infty \frac{1}{(3+(-1)^n)^n}$
    \end{center}

    \textbf{Answer:}
    \begin{itemize}
        \item $\limsup_{n\to\infty}\abs{\frac{(n+1)!}{(n+1)^{n+1}}/\frac{n!}{n^n}} = \limsup_{n\to\infty}\abs{\frac{n}{n+1}}^n = 1/e < 1 \Rightarrow$ converges
        \item $\limsup_{n\to\infty}\abs{\frac{1/(n+1)^2}{1/n^2}} = 1 \Rightarrow$ inconclusive
        \item $\limsup_{n\to\infty}\abs{\frac{(3+(-1)^n)^n}{(3+(-1)^{n+1})^{n+1}}} = \limsup_{n\to\infty}\abs{\frac{4^n}{2^{n+1}}} = 1 \Rightarrow$ inconclusive
    \end{itemize}

    \item  Wie lautet der kleine Umordnungssatz absolut konvergenter Reihen?
    
   
    \textbf{Answer:} Consider any $\sumn{n}{0}{\infty}{a_n}$ absolut convergent series, $\tau\colon \mathbb{N} \to \mathbb{N}$ permutation and define $b_n = a_{\tau^{-1}(n)}$. Then $\sumn{n}{0}{\infty}{b_n}$ is also absolutely convergent and $\sumn{n}{0}{\infty}{a_n} = \sumn{n}{0}{\infty}{b_n}$

    %Question 9
    \item  Wie lautet der große Umordnungssatz absolut konvergenter Reihen?
    
    \textbf{Answer:} Consider $a_{ij}:\NN^2 \to \KK$, $\tau:\NN^2 \to \NN$ bijection and let $b_n = a_{ij}$ for corresponding $i, j$ such that $n = \tau(i, j)$. Suppose furthermore that $\sumn{n}{0}{\infty}{b_n}$ converges absolutely. Then the series $\sigma_i = \sumn{j}{0}{\infty}{a_{ij}}~(\forall i\in\NN)$ and $s = \sumn{i}{0}{\infty}{\sigma_i} = \sumn{i}{0}{\infty}{\sumn{j}{0}{\infty}{a_{ij}}}$ converge, moreover they converge absolutely, and furthermore $\sumn{i}{0}{\infty}{\sigma_i} = s = \sumn{n}{0}{\infty}{b_n}$

    \item  Welche der folgenden Reihen konvergieren, welche konvergieren absolut?
    \begin{center}
        $\sum_{n=0}^\infty {\frac{1}{n}}$, $\sum_{n=0}^\infty \frac{(-1)^n}{n}$, $\sum_{n=0}^\infty \frac{(-1)^n}{n^2}$, $\sum_{n=0}^\infty \frac{x^n}{n!} \in \mathbb{C}$
    \end{center}

    \textbf{Answer:}
    \begin{itemize}
        \item $\sum_{n=1}^\infty {\frac{1}{n}}$ diverges (Verdichtungs-Kriterium)
        \item $\sum_{n=1}^\infty \frac{(-1)^n}{n}$ converges (Leibniz), but not absolutely, see previous point
        \item $\sum_{n=1}^\infty \frac{(-1)^n}{n^2}$ converges absolutely, since $\sum_{n=0}^\infty \frac{1}{n^2}$ converges (Verdichtungs-Kriterium)
        \item if $x\neq0$ then $\limsup_{n\to\infty}\abs{\frac{x^{n+1}/(n+1)!}{x^{n}/(n)!}} = \limsup_{n\to\infty}\abs{\frac{x}{n+1}} = 0 \Rightarrow$ converges (from Quotientenkriterium). If $x=0$ then it's converges trivially
    \end{itemize}
    \item  Für welche reellen/komplexen $s$ konvergiert die Reihe $\sum_{n=0}^\infty n^{-s}$ der Riemannschen
    $\zeta$-Funktion?

    \textbf{Answer:} Consider first $q\in \mathbb{R}: \sumn{n}{1}{\infty}{\frac{1}{n^q}} \Leftrightarrow \sumn{n}{0}{\infty}{\frac{2^n}{(2^n)^q}} = \sumn{n}{0}{\infty}{(2^n)^{1-q}} = \sumn{n}{0}{\infty}{(2^{1-q})^n}$ which converges for $1 < q \in \mathbb{R}$ (from geometric series) and diverges for $1 \le q \in \mathbb{R}$.

    Now for $q\in\CC, q = a + ib~(a, b\in\RR)\colon \abs{n^{-q}} =\abs{n^{-a}} \abs{(e^{-ib})^{\log{n}}} = \abs{n^{-a}}$, thus $\zeta(q)$ converges absolutely for $\Re(q) > 1$, and consequently it'll also converge conditionally.

    \item  Was ist eine Potenzreihe? Was ist ihr Konvergenzradius? Wie berechnet er sich?
    
    \textbf{Answer:} The formal powerseries centered in $c\in\CC$ with coefficients $(a_n) \in \CC$ is defined as the $p(x) = \sumn{n}{0}{\infty}{a_n(x-c)^n}$ series. The radius of convergence $\rho \in [0, +\infty)$ is defined as 

    $$\rho = 1/\limsup_{n\to\infty}\abs{a_n}^{1/n}$$

    If $\rho >0$ then we talk about power series. A $p(x)$ power series converges absolutely $\forall x\in\CC\colon \abs{x - c} < \rho$

    \item  Wann ist das Produkt zweier Potenzreihen wieder eine Potenzreihe? Wie lautet sie? Wie hängen die Konvergenzradien der Potenzreihen und ihres Produktes zusammen?
    
    \textbf{Answer: TODO}

    From the product theorem for absolutely convergent series we know that whenever two series are absolutely convergent, then so is their product where their product is such a $c_n$ sequence 



    \item  Wie lauten die Dastellungen von $\exp(x), \sin(x), \cos(x), \sinh(x), \cosh(x)$ als Potenzreihen?
    
    \textbf{Answer:}
    \begin{itemize}
        \item $\exp{z} = \sumn{n}{0}{\infty}{\frac{z^n}{n!}}$
        \item $\sin{z} = \sumn{k}{0}{\infty}{(-1)^k\frac{z^{2k+1}}{(2k+1)!}}$
        \item $\cos{z} = \sumn{k}{0}{\infty}{(-1)^k\frac{z^{2k}}{(2k)!}}$
        \item $\sinh{z} = \sumn{k}{0}{\infty}{\frac{x^{2k+1}}{(2k+1)!}}$
        \item $\cosh{z} = \sumn{k}{0}{\infty}{\frac{x^{2k}}{(2k)!}}$
    \end{itemize}

    \item  Wie hängen $e^z, \sin(z), \cos(z), \sinh(z), \cosh(z)$ im Komplexen zusammen?

    \begin{itemize}
        \item $e^{iz} = \cos{z} + i \sin{z}$
        \item $\sin{z} = \frac{e^{iz} - e^{-iz}}{2i}$
        \item $\cos{z} = \frac{e^{iz} + e^{-iz}}{2}$
        \item $\sinh{z} = \frac{e^{z} - e^{-z}}{2}$
        \item $\cosh{z} = \frac{e^{z} + e^{-z}}{2}$
        \item $\cosh{z} = \cos{iz}$
        \item $\sinh{z} = -i\sin{iz}$
    \end{itemize}
\end{enumerate}

\end{document} 
