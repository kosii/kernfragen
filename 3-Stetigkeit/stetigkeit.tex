\documentclass[11pt]{article}

\usepackage{amsmath,amssymb,amsfonts}
\usepackage{graphicx}
\usepackage[dvipsnames]{xcolor}
\usepackage{}


\newcommand{\norm}[1]{\left\lVert#1\right\rVert}
\newcommand{\abs}[1]{\left|#1\right|}

\setlength{\topmargin}{-.5in} \setlength{\textheight}{9.25in}
\setlength{\oddsidemargin}{0in} \setlength{\textwidth}{6.8in}


\begin{document}



\noindent{\bf Kernfragen - Stetigkeit\hfill Balázs Kossovics \hfill WS21-22 Analysis I.}


\medskip\hrule
\begin{enumerate}

\item Wann heißt eine Funktion $f$ in einem Punkt $x_0$ stetig? Welche äquivalente Definitionen der Stetigkeit gibt es (wenigstens drei verschiedene)?

$f$ is continuous in $x_0$ whenever one of the following equivalent conditions holds
\begin{itemize}
    \item $\forall (x_n) \in D\colon \lim_{n\to\infty}{x_n} = x \Rightarrow f(\lim_{n\to\infty}{x_n}) = f(x) = \lim_{n\to\infty}{f(x_n)}$
    \item $\forall \epsilon > 0\colon \exists \delta > 0\colon \forall x \in D, | x - x_0| < \delta\colon |f(x) - f(x_0)| < \epsilon$
    \item for any neighbourhood $V$ of $f(x_0)$ there is a neighbourhood $U$ of $x_0$ in $D$ such that $f(U) \subset V$
\end{itemize}

\item Wann heißt eine Funktion $f$ auf einer Menge $D \subseteq \mathbb{R}$ bzw. $D \subseteq \mathbb{C}$ stetig?

$f$ is continuous on the set $D$ whenever $f$ is continuous at all points $a \in D$

\item Sei $U = \{a_1, a_2, \dots, a_N \} \subset \mathbb{R}$ eine endliche Menge reeller Zahlen. Gib eine Funktion
$f\colon \mathbb{R} \to \mathbb{R}$ an, die auf $D = \mathbb{R} \setminus U$ stetig, auf $U$ aber unstetig ist.

$f(x) = \begin{cases}
    1& x \in U\\
    0&\text{otherwise}
\end{cases}$

Since any convergent $(x_n) \in \mathbb{R}$ will only contain at most finitely many element from $U$, and consequently if $x = \lim_{n \to \infty} x_n$, then $f(x) = f(\lim_{n \to \infty} x_n) = \lim_{n \to \infty} f(x_n) = \lim_{n \to \infty} 1 = 1 \Leftrightarrow x \notin U$

\item Gib eine Fuktion $f\colon \mathbb{R} \to \mathbb{R}$ an, die nirgends stetig ist.

$f(x) = \begin{cases}
    1&x\in \mathbb{Q}\\
    0&\text{otherwise}
\end{cases}$

\item Wie lautet der Zwischenwertsatz?

Consider any $D = [a, b] \subset R$ interval, and let $f\colon D \to \mathbb{R}$ be continuous. Then $f$ takes on any value in $[f(a), f(b)] \cup [f(b), f(a)]$

% question 6
\item Warum hat jede durch eine stetige Funktion $g\colon [0,1] \to [0,1]$ gegebene Iteration $x_{n+1} = g(x_n)$ (mindestens) einen Fixpunkt?

Consider $f(x) = g(x) - x$, $f(0) = g(0) - 0 \ge 0, f(1) = f(1) - 1 \le 0$. Thus from the intermediate value theorem $\exists c\in[0, 1]\colon f(c) = g(c) - c = 0 \Leftrightarrow g(c) = c$. If a so defined $(x_n) \in [0,1]$ converges, then it'll converge to the fixpoint:
$$f(\lim_{n\to\infty} x_n) = \lim_{n\to\infty} f(x_n) = \lim_{n\to\infty} x_{n+1}$$

But the convergence is not guaranteed, consider $g(x) = 1-x$ and $x_0 = 0$, which is divergent.


\item Wie lässt sich unter Benutzung des Zwischenwertsatzes zeigen, dass die Gleichung $\exp(x) = -x$ eine reelle Lösung besitzt?

\textbf{Answer:} Consider the $g(x) = e^x + x$ function on the $[-1, 0]$ interval. $g(-1) = e^{-1} - 1 = \frac{1}{e} - 1 < 1/2 - 1 < -1/2$ and $g(0) = e^{0} - 0 = 1 > 0$, thus from the intermediate value theorem there must be some $c \in [-1, 0]\colon g(c) = 0$.

\item Wann heißt eine Funktion $f\colon D \to \mathbb{R}, D \subseteq \mathbb{R}$, gleichmäßig stetig? Unter welcher (hinreichenden) Bedingung sind stetige Funktionen gleichmäßig stetig?

\textbf{Answer:} $f$ is uniformly continuous if $\forall \epsilon > 0\colon \exists \delta > 0\colon \forall x, y \in D\colon | x - y| < \delta \Rightarrow |f(x) - f(y) | < \epsilon$

If $D$ is closed and bounded, and $f$ is continuous on $D$, then $f$ is also uniformly continuous on $D$.

% question 9
\item  Welche dieser Funktionen $f\colon \mathbb{R} \to \mathbb{R}$ sind stetig, welche gleichmäßig stetig?
 \begin{center}
    $|x|$,   $\operatorname{exp}(x)$, $x^2$, $\sin(x)$, $\frac{x^3+1}{x^4-1}$, $\lceil x \rceil - x$
\end{center}
Hierbei bezeichnet die Gauß-Klammer, $\lceil x \rceil$, die größte ganze Zahl, die kleiner oder gleich $x$ ist.

\textbf{Answer:} 
\begin{enumerate}
    \item $|x|$ is continuous and furthermore is absolute continuous: consider $\epsilon > 0$ and some $x, y \in \mathbb{R}\colon |x - y| < \epsilon$ then $\epsilon > |x - y| > |x| - |y|$ and $\epsilon > |y - x| > |y| - |x|$ and thus $||x| - |y|| < \epsilon$. So $|f(x) - f(y)| = ||x| - |y|| < \epsilon$ and consequently it's absolute continuous.
    \item $\operatorname{exp}(x)$ is continuous, since it's defined by a powerseries with convergence radius $\rho = \infty$. A powerseries is continuous at every point inside it's convergence radius. It's not absolutely continuous: suppose indirectly that it was: $\forall \epsilon > 0\colon \exists \delta_\epsilon >0\colon \forall x, y \in \mathbb{R}: \abs{x - y} < \delta_\epsilon\colon \abs{f(x) - f(y)} < \epsilon$, or $\abs{e^x - e^y} < \epsilon$. This must hold in particular for $y = x + \delta_\epsilon/2\phantom{0}(\forall x \in \mathbb{R})$, and due to the strict monotonity of $e^x$ it also holds that $0 < e^{x+\delta/2} - e^x < \epsilon$ or equivalently $0 < e^x(e^{\delta/2} - 1) < \epsilon$. But $\lim_{x\to\infty} e^x(e^{\delta/2} - 1) = +\infty$, contradicion.
    
    \item $x^2$ is continuous, since $x$ is continuous, and since the multiple of two continuous function is continuous, thus so is $x^2$. On the other hand it's not uniformly continuous: suppose it was, that is $\forall \epsilon > 0\colon \exists \delta_\epsilon > 0\colon x, y \in \mathbb{R}\colon \abs{x - y} < \delta_\epsilon\Rightarrow\abs{x^2 - y^2} < \epsilon$. Consider now some $x > \delta_\epsilon > 0$ and $y = x + \delta_\epsilon/2$, the condition of uniform continuity must hold for these $x, y$ as well: 
        $$0 < (x+\delta_\epsilon/2)^2 - x^2 < \epsilon$$
    or
        \begin{equation}\label{eq:1}0 < x\delta_\epsilon + \delta_\epsilon^2/4 < \epsilon\end{equation}
    But since $\lim_{x\to\infty}x\delta_\epsilon + \delta_\epsilon^2/4 = \infty$, $\eqref{eq:1}$ cannot hold for every $x\in\mathbb{R}$.
    \item 
    $\sin$ is defined by a powerseries with a radius of convergence $\rho = \infty$ thus it's continuous on $\mathbb{R}$. Consider now $\sin$ on the closed and bounded interval $[0, 4\pi]$ ($4\pi$ was chosen here intentionally, so $\forall x, y \in \mathbb{R}$ at least one of  $x-2k\pi, y-2k\pi \in [0, 4\pi]$ or $x-4k\pi, y-4k\pi \in [0, 4\pi]$ will hold for some corresponding $k\in\mathbb{Z}$), where it's now uniformly continuous, thus $\forall \epsilon > 0\colon \exists \delta > 0\colon \forall x, y \in [0, 4\pi]\colon \abs{x - y} < \delta \Rightarrow \abs{\sin{x} - \sin{y}} < \epsilon$. $\sin$ is furthermore periodic with a period of $2\pi$, thus $\forall x, y \in \mathbb{R}\colon \abs{x-y} < \delta \Rightarrow \abs{\sin{x} - \sin{y}} < \epsilon$ will also hold.
    \item $x^3 + 1$ and $x^4-1$ are continuous on $\mathbb{R}$, so $f$ is continuous exactly when $x^4-1 \neq 0$, thus $f$ is continuous on $\mathbb{R}\setminus\left\{1\right\}$. On the other hand $\lim_{h\to1^-}f(1+h) = -\infty$ and $\lim_{h\to1^+}f(1+h) = +\infty$, so there will be $x < 1 < y$ arbitrarily close to each other such that $\abs{f(x) - f(y)}$ is arbitrarily large.
    \item $f(x) = x\phantom{0}(\forall x \in [0, 1))$ and $f$ has a periodicity of $1$, thus $\lim_{x\to1-} = 1$ but $\lim_{x\to1+} = 0$ thus $f$ is continuous on $\mathbb{R} \setminus \mathbb{Z}$. On the other hand $f$ is not uniformly continuous, because we can make arbitrarily close $3/4 < x < 1 < y < 5/4$ but $\abs{f(x) - f(y)} > 1/2$.
\end{enumerate}

\item Gib stetige Funktionen $f\colon D \to \mathbb{R}, D \subseteq \mathbb{R}$ an, die ihr Supremum annehmen, und solche, die ihr Supremum nicht annehmen. Unter welcher (hinreichenden) Bedingung nimmt eine stetige Funktion ihr Supremum an?

\textbf{Answer:} Let $D = [0, 1)$ and $f\colon D \to \mathbb{R}, f(x) = x^2$. $sup f = 1$, but $f$ does not take on it's supremum (because $f(x) = 1 \Leftrightarrow x = \pm1 \notin D$).

Let $D = [0, 1]$ and $f\colon D \to \mathbb{R}, f(x) = x^2$. $\sup{f} = \max{f} = 1$, thus $f$ takes on it's supremum.

If $D$ is bounded and closed, then $f\colon D \to \mathbb{R}$ takes on its supremum.

% Question 11
\item Sind die Bilder von Intervallen unter stetigen Abbildungen $f\colon \mathbb{R} \to \mathbb{R}$ wieder Intervalle? Sind stetige Bilder offener Intervalle wieder offene Intervalle?

\textbf{Answer:} Yes (\textbf{Why?}). The image of $f\colon (0, 1) \to \mathbb{R}$ with $f(x) = 1$ is a closed inteval $[1, 1]$.

\item Wo sind Potenzreihen stetig? Wo sind sie gleichmäßig stetig?

\textbf{Answer:} Consider $p(x)$ powerseries centered at $0$ with convergence radius of $\rho \in [0, \infty)$, and circle of convergence $C = \left\{x\colon |x| < \rho\right\}$. $p$ is continuous at each point of $C$. Consider any $D \subset C$ that is closed and bounded. Then $f$ is uniformly continuous on $D$. Contrary to the normal continuity, uniform continuity cannot be extended to the whole $C$ by considering a closed circle of radius $0 \le r < \rho$ inside $C$, and taking the limit $r \to \rho$.

\item Wann existiert die Inverse $f^{-1}$ einer stetigen Funktion $f\colon [a, b] \to \mathbb{R}$? Wann ist die Inverse stetig?

\textbf{Answer:} $f^{-1}$ exists exactly when $f$ is strictly monotonous. Whenever the inverse of a continuous function exists, it's always continuous.

\item Was ist ein normierter Vektorraum über $\mathbb{R}$ bzw. $\mathbb{C}$ ? Was ist ein Banachraum?

\textbf{Answer:} Consider $V$ vectorspace over $\mathbb{K}$. Then the $\norm{.}\colon V \to \mathbb{R}$ function is a norm, if it satisfies the following conditions ($\forall x, y \in V, \lambda \in \mathbb{K}$):
\begin{itemize}
    \item $\norm{x} \ge 0$ and $\norm{x} = 0 \Leftrightarrow x = 0$
    \item $\norm{\lambda x} = \abs{\lambda}\norm{x}$
    \item $\norm{x + y} \le \norm{x} + \norm{y}$
\end{itemize}

A normed vectorspace is complete, if every Cauchy-sequence is convergent (both property considered under the norm). A Banach-space is a complete normed vectorspace.

\item Was bedeutet Konvergenz in einem normierten Vektorraum?

\textbf{Answer:} Let $V$ is a vectorspace with norm $\norm{.}\colon V \to \mathbb{R}$. We say that $(x_n) \in V$ converges if $\exists v \in V\colon \lim_{n\to \infty}\norm{v_n - v} = 0$

\item Wie ist die Supremums-Norm für beschränkte, stetige Funktionen $f\colon D \to \mathbb{R}, D \subseteq \mathbb{R}$, definiert? Warum ist sie tatsächlich eine Norm?

\textbf{Answer:} $\norm{f}_{sup} = \sup\left\{|f(x)|\colon x \in D\right\}$. 

The above defined $\norm{.}_{sup}$ function satisfies the norm properties: 

$\forall f, g: D \to \mathbb{R}, \forall \lambda\in\mathbb{R}$:
\begin{itemize}
    \item $\norm{f} \ge 0$, $\norm{f} = 0 \Leftrightarrow f = 0$
    \item $\norm{\lambda f} = \sup\left\{|\lambda f(x)|\colon x \in D\right\} =  \sup\left\{|\lambda||f(x)|\colon x \in D\right\} \\=  |\lambda| \sup\left\{|f(x)|\colon x \in D\right\} = |\lambda| \norm{f}$
    \item $\norm{f + g} = \sup\left\{|f(x) + g(x)|\colon x \in D\right\} \le \sup\left\{|f(x)| + |g(x)|\colon x \in D\right\}\le \sup\left\{|f(x)| + \sup\{|g(y)|\colon y \in D \}\colon x \in D\right\} = \sup\left\{|f(x)| + \norm{g}\colon x \in D\right\} = \sup\left\{|f(x)| \colon x \in D\right\} + \norm{g} = \norm{f} + \norm{g}$
\end{itemize}

\item Warum ist der Raum der beschränkten, stetigen Funktionen, $\mathcal{BC}(D,\mathbb{R})$, mit der Supremums-Norm ein Banachraum?

\textbf{Answer:} The $\norm{.}_{\sup}$ is indeed a norm, as proved in the previous point. Now we only have to prove, that $\mathcal{BC}(D,\mathbb{R})$ is complete, that is: any $(f_n) \in\mathcal{BC}(D,\mathbb{R})$ Cauchy-sequence converges.

For $x\in D$ fixed: $$\abs{f_n(x) - f_m(x)} = \abs{(f_n - f_m)(x)} \le \sup\left\{\abs{(f_n -f_m)(x)}\colon x \in D\right\} = \norm{f_n - f_m}$$ Since $f_n$ is a Cauchy-sequence in norm, 
$$\forall \epsilon > 0\colon\exists N \in \mathbb{N}\colon n, m > N\colon \norm{f_n - f_m} < \epsilon$$
thus for such $n, m\colon \abs{f_n(x) - f_m(x)} < \epsilon$, thus $f_n$ converges pointwise.

Let $f(x) = \lim_{n\to\infty}f_n(x)$ be defined as the pointwise limit of $f_n$. It remains to prove that $\lim_{n\to\infty} \norm{f_n - f} = 0$ and that $f \in \mathcal{BC}(D,\mathbb{R})$.

Since $f_n$ is a Cauchy sequence, $\norm{f_n - f_m} < \epsilon$ holds for large enough $n,m$. It'll also hold for any $x\in D$ in particular that $\abs{f_n(x) - f_m(x)} < \epsilon$, and by taking $m$ to the limit we get $\abs{f_n(x) - f(x)} \le \epsilon$, so $\norm{f_n - f} \le \epsilon$ will hold as well, and thus $\lim_{n\to\infty}\norm{f_n - f} = 0$.

As for the continuity, we have to prove that $\forall a \in D\colon\forall \epsilon > 0\colon\exists\delta > 0\colon \abs{x - a} < \delta \Rightarrow \abs{f(x) - f(a)} < \epsilon$. 


$$\abs{f(x) - f(a)} \le \abs{f(x) - f_n(x)} + \abs{f_n(x) - f_n(a)} + \abs{f_n(a) - f(a)} < 3\epsilon$$

Because of the continuity of each $f_n\phantom{0}(n \in \mathbb{N})$ and because of the uniform convergence of $(f_n)$ sequence.

\item Gib ein Beispiel einer Funktionenfolge $f_n\colon [0,1] \to [0,1]$ an, die punktweise aber nicht gleichmäßig konvergiert.

\textbf{Answer:} $f_n(x) = x^n$

% Question 19
\item Auf welchen (möglichst großen) Intervallen konvergieren folgende Funktionenfolgen gleichmäßig?

\begin{center}
    $f_n(x) = \frac{1}{1 + n^2 x^2}$, $g_n(x) = \exp(-nx^2)$, $h_n(x) = \sum_{k=0}^n{(-1)^k x^k}$
\end{center}

\begin{itemize}
    \item $f_n \in \mathcal{BC}(\mathbb{R},\mathbb{R})$, but $\lim_{n\to\infty}f_n(0) = 1$ and $x\neq0\colon \lim_{n\to\infty}f_n(x) = 0$ thus $f(x) = \lim_{n\to\infty}f_n(x)$ is not continuous, $f \notin \mathcal{BC}(\mathbb{R},\mathbb{R})$, and consequently $f_n$ cannot converge uniformly on the whole $\mathbb{R}$, otherwise $f$ would have been continuous as well. Thus if there is any interval $I$ on which $f_n$ converge uniformly, it cannot contain $0$. For the same reason $I$ cannot have $0$ as its limit point, because $\lim_{x\to0}f_n(x) = 1$ and thus $\norm{f_n - f} = 1~(\forall n \in \mathbb{N})$ on that interval. On the other hand $\forall \delta > 0$ it converges uniformly on  $[\delta, \infty)$ and on $(-\infty, -\delta]$, because $\norm{f_n - f} = \norm{f_n} = \frac{1}{1+n^2\delta^2} \to 0~(n\to\infty)$
    \item $g_n \in \mathcal{BC}(\mathbb{R},\mathbb{R})$ and $g_n(0) = 1$ but $\forall x\neq0\colon \lim_{n\to\infty}g_n(x) = 0$, thus $g(x) = \lim_{n\to\infty}g_n(x)$ is not continuous, thus $g_n$ cannot converge on the whole $\mathbb{R}$. For similar reasons as in the previous point, $g_n$ will converge uniformly on  $[\delta, \infty)$ and on $(-\infty, -\delta]~(\delta > 0)$.
    \item It's a powerseries with a convergence radius of $$\rho =1/\limsup_{n\to\infty}\abs{(-1)^n}^{1/n} =~1$$ thus it'll converge on $(-1, 1)$. It'll furthermore converge unformly on $[-r, r]~(\forall\rho > r > 0)$ (from theorem in 3.9 of Continuity).
\end{itemize}


\item Was ist die Umkehrfunktion von $\exp(x)$. Welche Funktionalgleichung erfüllt sie? Wo
ist sie definiert? Wo ist sie stetig?

\textbf{Answer:} Since $\exp$ is strictly monotonous and continuous, it's inverse exists and it's also continuous at its respective domain of definition: $$\log\colon \mathbb{R}^+ \to \mathbb{R}, \log := \exp^{-1}$$
Since it's the inverse of a continuous function, it's continuous everywhere.
\item Wie ist die allgemeine Potenz $x^\alpha$ für $\alpha \in \mathbb{C}$ und $x \in \mathbb{R}^+$ definiert?

\textbf{Answer:} $x^\alpha = e^{\log{x} \alpha}$

\end{enumerate}

\end{document} 
