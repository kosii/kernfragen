\documentclass[11pt]{article}

\usepackage{amsmath,amssymb,amsfonts}
\usepackage{graphicx}
\usepackage[dvipsnames]{xcolor}
\usepackage{}


\newcommand{\norm}[1]{\left\lVert#1\right\rVert}
\newcommand{\abs}[1]{\left|#1\right|}

\setlength{\topmargin}{-.5in} \setlength{\textheight}{9.25in}
\setlength{\oddsidemargin}{0in} \setlength{\textwidth}{6.8in}


\begin{document}

\Large


\noindent{\bf Kernfragen - Stetigkeit\hfill WS21-22 Analysis I.}


\medskip\hrule
\begin{enumerate}

\item Wann heißt eine Funktion $f$ in einem Punkt $x_0$ stetig? Welche äquivalente Definitionen der Stetigkeit gibt es (wenigstens drei verschiedene)?

$f$ is continuous in $x_0$ whenever one of the following equivalent conditions holds
\begin{itemize}
    \item $\forall (x_n) \in D\colon \lim_{n\to\infty}{x_n} = x \Rightarrow f(\lim_{n\to\infty}{x_n}) = f(x) = \lim_{n\to\infty}{f(x_n)}$
    \item $\forall \epsilon > 0\colon \exists \delta > 0\colon \forall x \in D, | x - x_0| < \delta\colon |f(x) - f(x_0)| < \epsilon$
    \item for any neighbourhood $V$ of $f(x_0)$ there is a neighbourhood $U$ of $x_0$ in $D$ such that $f(U) \subset V$
\end{itemize}

\item Wann heißt eine Funktion $f$ auf einer Menge $D \subseteq \mathbb{R}$ bzw. $D \subseteq \mathbb{C}$ stetig?

$f$ is continuous on the set $D$ whenever $f$ is continuous at all points $a \in D$

\item \textbf{TODO} Sei $U = \{a_1, a_2, \dots, a_N \} \subset \mathbb{R}$ eine endliche Menge reeller Zahlen. Gib eine Funktion
$f\colon \mathbb{R} \to \mathbb{R}$ an, die auf $D = \mathbb{R} \setminus U$ stetig, auf $U$ aber unstetig ist.

$f(x) = \begin{cases}
    1& x \in U\\
    0&\text{otherwise}
\end{cases}$

Since any convergent $(x_n) \in \mathbb{R}$ will only contain at most finitely many element from $U$, and consequently if $x = \lim_{n \to \infty} x_n$, then $\lim_{n \to \infty} x_n$

\item Gib eine Fuktion $f\colon \mathbb{R} \to \mathbb{R}$ an, die nirgends stetig ist.

$f(x) = \begin{cases}
    1&x\in \mathbb{Q}\\
    0&\text{otherwise}
\end{cases}$

\item Wie lautet der Zwischenwertsatz?

Consider any $D = [a, b] \subset R$ interval, and let $f\colon D \to \mathbb{R}$ be continuous. Then $f$ takes on any value in $[f(a), f(b)] \cup [f(b), f(a)]$

\item \textbf{TODO} Warum hat jede durch eine stetige Funktion $g\colon [0,1] \to [0,1]$ gegebene Iteration $x_{n+1} = g(x_n)$ (mindestens) einen Fixpunkt?




\item Wie lässt sich unter Benutzung des Zwischenwertsatzes zeigen, dass die Gleichung $\exp(x) = -x$ eine reelle Lösung besitzt?

\textbf{Answer:} Consider the $g(x) = e^x - x$ function on the $[1/e, 1]$ interval. $g(1/e) = e^{1/e} - e < e^1 - e = 0$ and $g(1) = e^1 - 1 > 2 - 1 > 0$, thus from the intermediate value theorem there must be some $c \in [1/e, 1]\colon g(c) = 0$.

\item Wann heißt eine Funktion $f\colon D \to \mathbb{R}, D \subseteq \mathbb{R}$, gleichmäßig stetig? Unter welcher (hinreichenden) Bedingung sind stetige Funktionen gleichmäßig stetig?

\textbf{Answer:} $f$ is uniformly continuous if $\forall \epsilon > 0\colon \exists \delta > 0\colon \forall x, y \in D\colon | x - y| < \delta \Rightarrow |f(x) - f(y) | < \epsilon$

If $D$ is closed and bounded, and $f$ is continuous on $D$, then $f$ is also uniformly continuous on $D$.

\item \textbf{TODO}  Welche dieser Funktionen $f\colon \mathbb{R} \to \mathbb{R}$ sind stetig, welche gleichmäßig stetig?
 \begin{center}
    $|x|$,   $\operatorname{exp}(x)$, $x^2$, $\sin(x)$, $\frac{x^3+1}{x^4-1}$, $\lceil x \rceil - x$
\end{center}
Hierbei bezeichnet die Gauß-Klammer, $\lceil x \rceil$, die größte ganze Zahl, die kleiner oder gleich $x$ ist.

\textbf{Answer:} 
\begin{enumerate}
    \item $|x|$ is continuous and furthermore is absolute continuous: consider $\epsilon > 0$ and some $x, y \in \mathbb{R}\colon |x - y| < \epsilon$ then $\epsilon > |x - y| > |x| - |y|$ and $\epsilon > |y - x| > |y| - |x|$ and thus $||x| - |y|| < \epsilon$. So $|f(x) - f(y)| = ||x| - |y|| < \epsilon$ and consequently it's absolute continuous.
    \item $\operatorname{exp}(x)$ is continuous, since it's defined by a powerseries with convergence radius $\rho = \infty$. A powerseries is continuous at every point inside it's convergence radius. It's not absolutely continuous: consider some $\epsilon > 0$ and any point $a \in \mathbb{R}$. $\exp(a+h) \in (e^a-\epsilon, e^a + \epsilon)$
    
    \item $x^2$ is continuous, since $x$ is continuous, and since the multiple of two continuous function is continuous, thus so is $x^2$. On the other hand it's not 
\end{enumerate}

\item Gib stetige Funktionen $f\colon D \to \mathbb{R}, D \subseteq \mathbb{R}$ an, die ihr Supremum annehmen, und solche, die ihr Supremum nicht annehmen. Unter welcher (hinreichenden) Bedingung nimmt eine stetige Funktion ihr Supremum an?

\textbf{Answer:} Let $D = [0, 1)$ and $f\colon D \to \mathbb{R}, f(x) = x^2$. $sup f = 1$, but $f$ does not take on it's supremum (because $f(x) = 1 \Leftrightarrow x = \pm1 \notin D$).

Let $D = [0, 1]$ and $f\colon D \to \mathbb{R}, f(x) = x^2$. $\sup{f} = \max{f} = 1$, thus $f$ takes on it's supremum.

If $D$ is bounded and closed, then $f\colon D \to \mathbb{R}$ takes on its supremum.

\item \textbf{Answer:} Sind die Bilder von Intervallen unter stetigen Abbildungen $f\colon \mathbb{R} \to \mathbb{R}$ wieder Intervalle? Sind stetige Bilder offener Intervalle wieder offene Intervalle?

\textbf{Answer:} Yes (\textbf{Why?}). The image of $f\colon (0, 1) \to \mathbb{}R$ with $f(x) = 1$ is a closed inteval $[1, 1]$.

\item Wo sind Potenzreihen stetig? Wo sind sie gleichmäßig stetig?

\textbf{Answer:} Consider $p(x)$ powerseries centered at $0$ with convergence radius of $\rho \in [0, \infty)$, and circle of convergence $C = \left\{x\colon |x| < \rho\right\}$. $p$ is continuous at each point of $C$. Consider any $D \subset C$ that is closed and bounded. Then $f$ is uniformly continuous on $D$. Contrary to the normal continuity, uniform continuity cannot be extended to the whole $C$ by considering a closed circle of radius $0 \le r < \rho$ inside $C$, and taking the limit $r \to \rho$.

\item Wann existiert die Inverse $f^{-1}$ einer stetigen Funktion $f\colon [a, b] \to \mathbb{R}$? Wann ist die Inverse stetig?

\textbf{Answer:} $f^{-1}$ exists exactly when $f$ is strictly monotonous. Whenever the inverse of a continuous function exists, it's always continuous.

\item Was ist ein normierter Vektorraum über $\mathbb{R}$ bzw. $\mathbb{C}$ ? Was ist ein Banachraum?

\textbf{Answer:} Consider $V$ vectorspace over $\mathbb{K}$. Then the $\norm{.}\colon V \to \mathbb{R}$ function is a norm, if it satisfies the following conditions ($\forall x, y \in V, \lambda \in \mathbb{K}$):
\begin{itemize}
    \item $\norm{x} \ge 0$ and $\norm{x} = 0 \Leftrightarrow x = 0$
    \item $\norm{\lambda x} = \abs{\lambda}\norm{x}$
    \item $\norm{x + y} \le \norm{x} + \norm{y}$
\end{itemize}

A normed vectorspace is complete, if every Cauchy-sequence is convergent (both property considered under the norm). A Banach-space is a complete normed vectorspace.

\item Was bedeutet Konvergenz in einem normierten Vektorraum?

\textbf{Answer:} Let $V$ is a vectorspace with norm $\norm{.}\colon V \to \mathbb{R}$. We say that $(x_n) \in V$ converges if $\exists v \in V\colon \lim_{n\to \infty}\norm{v_n - v} = 0$

\item Wie ist die Supremums-Norm für beschränkte, stetige Funktionen $f\colon D \to \mathbb{R}, D \subseteq \mathbb{R}$, definiert? Warum ist sie tatsächlich eine Norm?

\textbf{Answer:} $\norm{f}_{sup} = \sup\left\{|f(x)|\colon x \in D\right\}$. 

The above defined $\norm{.}_{sup}$ function satisfies the norm properties: 

$\forall f, g: D \to \mathbb{R}, \forall \lambda\in\mathbb{R}$:
\begin{itemize}
    \item $\norm{f} \ge 0$, $\norm{f} = 0 \Leftrightarrow f = 0$
    \item $\norm{\lambda f} = \sup\left\{|\lambda f(x)|\colon x \in D\right\} =  \sup\left\{|\lambda||f(x)|\colon x \in D\right\} \\=  |\lambda| \sup\left\{|f(x)|\colon x \in D\right\} = |\lambda| \norm{f}$
    \item $\norm{f + g} = \sup\left\{|f(x) + g(x)|\colon x \in D\right\} \le \sup\left\{|f(x)| + |g(x)|\colon x \in D\right\}\le \sup\left\{|f(x)| + \sup\{|g(y)|\colon y \in D \}\colon x \in D\right\} = \sup\left\{|f(x)| + \norm{g}\colon x \in D\right\} = \sup\left\{|f(x)| \colon x \in D\right\} + \norm{g} = \norm{f} + \norm{g}$
\end{itemize}

\item \textbf{TODO}Warum ist der Raum der beschränkten, stetigen Funktionen, $\mathcal{BC}(D,\mathbb{R})$, mit der Supremums-Norm ein Banachraum?

\textbf{Answer:}

\item Gib ein Beispiel einer Funktionenfolge $f_n\colon [0,1] \to [0,1]$ an, die punktweise aber nicht gleichmäßig konvergiert.

\textbf{Answer:} $f_n(x) = x^n$
\item Auf welchen (möglichst großen) Intervallen konvergieren folgende Funktionenfolgen gleichmäßig?

\begin{center}
    $f_n(x) = \frac{1}{1 + n^2 x^2}$, $f_n(x) = \exp(-nx^2)$, $f_n(x) = \sum_{k=0}^n{(-1)^k x^k}$
\end{center}

\textbf{Answer:}
\item Was ist die Umkehrfunktion von $\exp(x)$. Welche Funktionalgleichung erfüllt sie? Wo
ist sie definiert? Wo ist sie stetig?

\textbf{Answer:} Since $\exp$ is strictly monotonous and continuous, it's inverse exists and it's also continuous at its respective domain of definition: $$\log\colon \mathbb{R}^+ \to \mathbb{R}, \log := \exp^{-1}$$
Since it's the inverse of a continuous function, it's continuous everywhere.
\item Wie ist die allgemeine Potenz $x^\alpha$ für $\alpha \in \mathbb{C}$ und $x \in \mathbb{R}^+$ definiert?

\textbf{Answer:} $x^\alpha = e^{\log{x} \alpha}$

\end{enumerate}

\end{document} 
