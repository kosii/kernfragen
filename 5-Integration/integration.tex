\documentclass[11pt]{article}

\usepackage{amsmath,amssymb,amsfonts}
\usepackage{graphicx}
\usepackage{mathtools}
\usepackage{bbm}
\usepackage[dvipsnames]{xcolor}
\usepackage{}


\newcommand{\norm}[1]{\left\lVert#1\right\rVert}
\newcommand{\abs}[1]{\left|#1\right|}
\newcommand{\sumn}[4]{\sum_{#1=#2}^{#3}{#4}}
\newcommand{\RR}[0]{\mathbb{R}}
\newcommand{\CC}[0]{\mathbb{C}}
\newcommand{\QQ}[0]{\mathbb{Q}}
\newcommand{\ZZ}[0]{\mathbb{Z}}
\newcommand{\NN}[0]{\mathbb{N}}
\newcommand{\KK}[0]{\mathbb{K}}
\newcommand{\smallo}[0]{{\scriptstyle \mathcal{O}}}
\DeclarePairedDelimiter\floor{\lfloor}{\rfloor}
\newcommand{\slim}[2]{\lim_{#1\to\infty}{#2}}
\renewcommand{\Re}[0]{\operatorname{Re}}
\renewcommand{\Im}[0]{\operatorname{Im}}

\setlength{\topmargin}{-.5in} \setlength{\textheight}{9.25in}
\setlength{\oddsidemargin}{0in} \setlength{\textwidth}{6.8in}


\begin{document}

\noindent{\bf Kernfragen - Integration\hfill Balázs Kossovics \hfill 2022 SS - Analysis II.}


\medskip\hrule
\begin{enumerate}
    \item Was ist ein normierter Raum? Wann sagt man, dass ein normierter Raum Banach ist?

    \textbf{Answer:} Let $V$ be a vector space over $\mathbb{K}$ ($\mathbb{K} = \mathbb{R}$ or $\mathbb{C}$) and $\norm{.}\colon V \to \mathbb{R}$. The pair $(V, \norm{.})$ called a normed vectorspace if $\norm{.}$ satisfies the following properties:
    \begin{enumerate}
        \item $\forall v \in V\colon \norm{v} \ge 0$ and $\norm{v} = 0 \Leftrightarrow v = 0_V$
        \item $\forall v \in V, \lambda \in \mathbb{K}\colon \norm{\lambda v} = \abs{\lambda} \norm{v}$
        \item $\forall u, v \in V\colon \norm{u + v} \le \norm{u} + \norm{v}$
    \end{enumerate}
    We call furthermore $(V, \norm{.})$ normed space a Banach space, if it's complete, that is: every Cauchy sequence has a limit in $V$.

    \item Wann sagt man, dass eine Funktion zwischen zwei normierten Räumen stetig ist?

    \textbf{Answer:} Condider $(U, \norm{.}_U)$ and $(V, \norm{.}_V)$ normed spaces, $A \subset U$ and a function $f\colon A \to V$. $f$ is continuous in $a \in A$, if $\forall \varepsilon > 0\colon \exists \delta > 0\colon \forall x\in A\colon \norm{x - a}_U < \delta \Rightarrow \norm{f(x) - f(a)}_V < \varepsilon$. $f$ is continuous on $A$, if it's continuous in every point of $A$.

    \item Seien $X$ und $Z$ normierte Räume. Was ist die Operatornorm $\norm{L}$ einer linearen Abbildung $L\colon X \to Z$? Was kann man über $\norm{L}$ sagen, wenn $L$ stetig ist?

    \textbf{Answer:} $\norm{L} = \sup_{\norm{x}_X = 1} \norm{L x}_Z = \sup_{0\neq x \in X}\frac{\norm{L x}_Z}{\norm{x}_X}$. Linear operators are continuous if and only if they are bounded.

    \item Was ist eine Regelfunktion? Welche äquivalenten Charakteriesierungen gibt es (wenigstens 2)?

    \textbf{Answer:}
    Let $V$ be a Banach space (including $\CC$ or $\RR$). The set $\mathcal{R}([a, b], V)$ of regulated functions is the closure of the set $\mathcal{T}([a, b], V)$ of step functions with regards to the set $B([a, b], V)$ of bounded functions under the $\norm{.}_{\sup}$ norm. Equivalently: \begin{enumerate}
    \item $f \in B([a, b], V)$ is a regulated function $\Leftrightarrow \forall c \in [a, b]\colon \exists \lim_{x\to c^+} f(x), \lim_{x\to c^-} f(x)$
    \item $f \in B([a, b], V)$ is a regulated function $\Leftrightarrow \exists (f_n) \in \mathcal{T}([a, b], V)$ uniformly convergent sequence of stepfunctions such that $\lim_{n \to \infty} f_n = f$ (in $B([a, b], V)$)
    \end{enumerate}

    \item Gib je zwei Beispiele an für
    \begin{enumerate}
        \item Regelfunktionen und
        \item Funktionen, die keine Regelfunktionen sind.
    \end{enumerate}

    \textbf{Answer:}
    \begin{enumerate}
        \item Regelfunktionen:\begin{enumerate}
            \item $f: [0, 1] \to \mathbb{R}, x \mapsto 1$
            \item $f: [-1, 1] \to \mathbb{R}, x \mapsto \begin{cases}
                x^2&x\neq0\\
                1&x=0
            \end{cases}$
        \end{enumerate}

        \item Keine Regelfunktionen:\begin{enumerate}
            \item $\mathbbm{1}_\QQ\colon [0, 1] \to \RR, x \mapsto \begin{cases}
                1&x\in\QQ\\
                0&x\notin\QQ
            \end{cases}$
            \item $f\colon [0, 1]\to \RR, x \mapsto \begin{cases}
                \sin{\frac{1}{x}}&x\neq 0\\
                0&x = 0
            \end{cases}$
        \end{enumerate}
    \end{enumerate}
    \item Wie ist das Integral einer Regelfunktion definiert?

    \textbf{Answer:} Let $V$ be a Banach space over $\KK = \RR$ or $\CC$. Consider any $f \in \mathcal{R}([a, b], V)$ regulated function, and $f_n \in \mathcal{T}([a, b], V)$ sequence of step functions, that converge uniformly to $f$. Let furthermore $P_n = \left\{p_0, \dots, p_{k_n}\right\}$ be such a partition of $[a,b]$ for which $f_n\bigm|_{[p_i, p_{i+1}]} = c_i \in V$ constant (with the potential exception of the endpoints). Let the $\int_a^b: \mathcal{T}([a, b], V) \to V$ linear operator be defined as $\int_a^b f_n = \sum_{i=0}^{k_n} c_i (p_{i+1} - p_i)$. Since $\mathcal{T}([a, b], V)$ is a subspace of $\mathcal{R}([a, b], V)$, there exists a unique continuous continuation $\overline{\int_a^b}\colon \mathcal{R}([a, b], V) \to V$ of the linear operator $\int_a^b$ such that their values stays the same on $\mathcal{T}([a, b], V)$. Since $\overline{\int_a^b}$ is continuous, it "commutes" with the limit. Let thus $\int_a^b f:=\overline{\int_a^b}f = \overline{\int_a^b} \lim_{n\to \infty} f_n = \lim_{n\to\infty} \int_a^b f_n$

    \item Wie hängen Integration und Differentiation zusammen?

    \textbf{Answer:} Let $V$ be a Banach space over $\KK = \RR$ or $\CC$, and $f\colon [a, b] \to V$ function. Let furthermore $F(x) = \int_a^x f$. If $f$ is continuous in $c \in [a, b]$, then $F$ is differentiable in $c$, and $F^\prime(c) = f(c)$. Furthermore if $f$ is continuous, and $F$ is such a function, that $F^\prime = f$, then $\int_a^b f = F(b) - F(a)$, and we call $F$ the primitive function of $f$.

    \item Welchen elementaren Funktionen entsprechen folgende unbestimmte Integrale?

    \hspace*{\fill}
    $\int{\sin{t}\,dt} \hfill \int{\frac{dt}{t}} \hfill \int{\sqrt[n]{t+1}} \hfill \int{\frac{1}{1+t^2}\,dt} \hfill \int{t^\alpha \,dt}~(\alpha \neq -1)$
    \hspace*{\fill}


    \textbf{Answer:}
    \begin{itemize}
        \item $\int{\sin{t}\,dt} = \cos{t}$
        \item $\int{\frac{dt}{t}} = \log{t}$
        \item $\int{\sqrt[n]{t+1}} = \frac{n}{n+1} (t+1)^{1 + \frac{1}{n}}$
        \item $\int{\frac{1}{1+t^2}\,dt} = \arctan{t}$
        \item $\int{t^\alpha \,dt} = \frac{1}{\alpha + 1} t^{\alpha + 1}~(\alpha \neq 1)$
    \end{itemize}
    \item Wie lauten die Regeln für partielle Integration und Substitution? Gib außerdem jeweils ein nichttriviales Beispiel an.

    \textbf{Answer:}

    \textit{Partial Integration}: Consider $f, g \in C^1([a, b], \KK)$ with $\KK = \RR$ or $\CC$. Then $\int_a^b f^\prime(t) g(t)\,dt = f(t) g(t)\bigm|_a^b - \int_a^b f(t) g^\prime(t)\,dt$

    Example: $\int \log{x} \,dx = x \log{x} - \int x \frac{1}{x} \,dx = x \log{x} - x$

    \textit{Substitution}: Consider $[a, b] \subset I_1, I_2$ intervals, $Z$ Banach space, and $f: I_2 \to Z$ continuous, and $g\colon I_1 \to I_2$ continuously differentiable. Then $\int_{g(a)}^{g(b)} f(t)\,dt = \int_a^b f(g(t)) g^\prime(t) \,dt$

    Example: Consider $\int_a^b \tan{x} \,dx$ and let $f(x) = \frac{1}{x}, g(x) = \cos{x}$. Then $\int_a^b \tan{x} \,dx = \int_a^b \tan{x}\,dx = \int_a^b \frac{\sin{x}}{\cos{x}}\,dx = \int_a^b f(g(x)) g^\prime(x) \,dx = \int_{g(a)}^{f(a)} f(x) \,dx = \int_{\cos(a)}^{\cos(b)} \frac{1}{x}\,dx = \log{x}\bigm|_{\cos(a)}^{\cos(b)} = \log(\cos(b)) - \log(\cos(a))$

    \item Wie integriert man rationale Funktionen? Welche elementaren Integrale muss man dazu kennen?

    \textbf{Answer: TODO} Consider $p, q \in \KK\left[x\right]$ polynomials over $\KK$ with $\KK = \RR$ or $\CC$.

    %11
    \item Was sagt das Riemann-Lebesgue Lemma? Skizziere einen Beweis für das Lemma.

    \textbf{Answer:} Let $f\colon [a, b] \to \RR$ continuously differentiable. Then $\lim_{\abs{\omega}\to\infty} \int_a^b f(t) \sin(\omega t) \,dt = 0$.
    \textit{Proof:} $$\begin{aligned}
        \abs{\int_a^b f(t) \sin{(\omega t)}\,dt} &= \abs{-\frac{\cos{(\omega t)}}{\omega}f(t)\Bigm|_a^b + \frac{1}{\omega} \int_a^b \cos{(\omega t)} f^\prime(t)\,dt} \\
        &< \abs{\frac{1}{\omega}} \left(\abs{\cos{(\omega b)} f(b)} + \abs{\cos{(\omega a)} f(a)} + \int_a^b \abs{\cos{(\omega t)} f^\prime(t)}\,dt \right)\\
        &\stackrel{\cos\text{ bounded}}{<} \abs{\frac{1}{\omega}} \left(\abs{\cos{(\omega b)} f(b)} + \abs{\cos{(\omega a)} f(a)} + \int_a^b \abs{f^\prime(t)}\,dt \right) \stackrel{\abs{\omega} \to \infty}{\to} 0
    \end{aligned}$$
    \item Wann dürfen Regelintegral und Grenzwert einer Funktionenfolge vertauscht werden?

    \textbf{Answer:} Consider $f_n \in \mathcal{R}([a, b], V)$ with $V$ Banach space. If $f_n$ converge uniformly to some $f\in \mathcal{R}([a, b], V)$, then $\int_a^b f = \lim_{n \to \infty} \int_a^b f_n$

    \item Wie ist das Riemann-Integral definiert?

    \textbf{Answer:} Consider the interval $[a, b]$, a function $f\colon [a, b] \to \KK$ (with $\KK = \RR$ or $\CC$), and for any partition $P = \left\{a = p_0 < p_1 < \dots < p_n = b\right\}$ of $[a, b]$ let $L(f, P) = \sum_{i = 1}^n \inf_{x \in (p_{i-1}, p_i)}f(x) (p_i - p_{i-1})$ and $U(f, P) = \sum_{i = 1}^n \sup_{x \in (p_{i-1}, p_i)}f(x) (p_i - p_{i-1})$. Let furthermore $U^*(f) = \inf_{P} U(f, P)$ and $L^*(f) = \sup_{P} L(f, P)$. If $U^*(f) = L^*(f)$, then we say that $f$ is Riemann integrable, and $\int_a^b f = U^*(f) = L^*(f)$.

    \item Was ist eine Lebesgue-Nullmenge?

    \textbf{Answer:} $A \subset \RR$ is a null set, if $\forall \epsilon > 0$ there is an at most countable collection of open intervals $I = \left\{I_i\right\}_{i \in \NN}$ such that $A \in \cup_{I_i \in I} I_i$ and $\sum_{I_i \in I} \abs{I_i} < \epsilon$

    \item Wie ist das Lebesgue-Integral definiert?


    \textbf{Answer:}

    \item Gib je ein Beispiel für eine Funktion an, die
    \begin{enumerate}
        \item Riemann integrierbar ist, aber ist keine Regelfunktion.
        \item Lebesgue integrierbar ist, aber nicht Riemann integrierbar.
    \end{enumerate}

    \textbf{Answer:}
    \begin{enumerate}
        \item $f\colon [0, 1] \to \RR, x \mapsto \begin{cases}
            \sin{\frac{1}{x}}&x \in (0, 1)\\
            0&x=0
        \end{cases}$
        \item The Dirichlet function
    \end{enumerate}

    \item Was sagt der Satz von Beppo Levi über monotone Folgen Lebesgue-integrierbarer Funktionen?

    \textbf{Answer:}
    \item Was sagt der Satz zur majorisierten Konvergenz über die Vertauschbarkeit von Lebesgue-Integral und Grenzwert einer Funktionenfolge?

    \textbf{Answer:}
    \item Wie lautet die Hölder-Ungleichung für integrierbare Funktionen?

    \textbf{Answer:}

    \item Wie lautet der Schrankensatz? Warum gilt der Mittelwertsatz (der Differentialrechnung) nicht in höheren Dimensionen?

    \textbf{Answer:}

    Let $Z$ Banach space and $f \in C^1([a, b], Z)$. Then $\exists \zeta \in (a, b)\colon \norm{f(b) - f(a)} \le \norm{f^\prime(\zeta)}(b-a)$.
    The mean value theorem of $f(b) - f(a) = f^\prime(\zeta)(b-a)$ does not hold in general. Consider $Z = \RR^3$ and $f(t) := \left(\begin{array}[]{c}
        \cos{t}\\\sin{t}\\\varepsilon t
    \end{array}\right)$ with some $\varepsilon > 0$. Then $f^\prime(t) =\left(\begin{array}[]{c}
        -\sin{t}\\\cos{t}\\\varepsilon
    \end{array}\right)$. Now if we consider the $[0, 2\pi]$ interval, then $f(2\pi k) - f(0) = 2\pi \left(\begin{array}[]{c}
        0\\0\\\varepsilon
    \end{array}\right) \neq 2\pi f^\prime(\zeta)~(\forall \zeta \in [0, 2\pi])$

    \item Wie lautet die Integraldarstellung von Lagrange für das Restglied der Taylorentwicklung?

    \textbf{Answer:} Let $I = (a, b) \subset \RR$ open interval, $n \in \NN_0, Z$ Banach space and $f \in C^{n+1}(I, Z)$. If $x_0 \in (a, b)$, then $\forall x \in (a, b)\colon f(x) = \sum_{k=0}^n \frac{1}{k!}f^{(k)}(x_0) (x - x_0)^k + \frac{1}{n!} \int_{x_0}^x (x-t)^n f^{(n+1)}(t)\,dt$
    \item Wie lautet die Trapezregel?

    \textbf{Answer:} Consider $f \in C^2([0, 1], \mathbb{R})$. Then $\exists \zeta \in (0, 1)\colon \int_0^1 f = \frac{1}{2}(f(0) + f(1))- \frac{1}{12}f^{\prime\prime}(\zeta)$

    \item Wie lassen sich Integrale dank der Trapezregel approximieren?

    \textbf{Answer:} Let $f \in C^2([a, b], \RR), C := \max_{x\in[a, b]}\abs{f^{\prime\prime}}, n \in \NN, h := \frac{b-a}{n}$. Then it holds that $$\abs{\int_a^b f - \left(\frac{1}{2}f(a_0) + \sum_{k=1}^{n-1}f(a_k) + \frac{1}{2}f(a_n)\right)h}\le \frac{1}{12}C(b-a)h^2$$
    \item Was sind uneigentliche Integrale?

    \textbf{Answer:} Let $-\infty \le a < b \le +\infty$ and $f\colon(a, b) \to \RR$ integrable over every finite $[\alpha, \beta] \subset (a, b)$ interval. Consider furthermore $(a_n), (b_n) \in \RR$ respectively monotone decreasing and increasing sequences such that $a < a_n$ and $b_n < b~(\forall n \in \NN)$ and suppose furthermore that $\lim_{n\to\infty}a_n = a$ and $\lim_{n\to\infty}b_n = b$. If for every such sequence the limit $\lim_{n\to\infty}\int_{a_n}^{b_n} f$ exists and has the same value, then let $\int_a^b f = \lim_{n\to\infty}\int_{a_n}^{b_n} f$ and call it the improper integral of $f$ over $(a, b)$.

    \item Für welche reellen Exponenten $\alpha$ konvergiert das uneigentliche Integral $\int_0^1t^\alpha\,dt$, für welche das uneigentliche Integral $\int_1^\infty t^\alpha \,dt$?

    \textbf{Answer:}
    \begin{itemize}
        \item $\int_0^1t^\alpha\,dt = \frac{1}{\alpha+1}~(-1 < \alpha \in \RR)$
        \item $\int_1^\infty t^\alpha \,dt = -\frac{1}{q+1}~(-1 > q \in \RR)$
    \end{itemize}

    \item Was bedeutet absolute Konvergenz uneigentlicher Integrale? Gib Beispiele
    \begin{enumerate}
        \item absolut konvergenter;
        \item konvergenter, aber nicht absolut konvergenter;
        \item nicht konvergenter
    \end{enumerate} uneigentlicher Integrale.

    \textbf{Answer:} Consider the improper integral $\int_a^b f$. If $\int_a^b\abs{f}$ converges, then so is $\int_a^b f$, and we call the improper integral absolulte convergent.

    \begin{enumerate}
        \item absolute convergent: $\int_0^1 \sin{\frac{1}{t}}\,dt$
        \item convergent, but not absolutely: $\int_0^{+\infty}\frac{1}{t}\sin{t}\,dt$
        \item not convergent: $\int_1^{+\infty} \frac{1}{t}\,dt$
    \end{enumerate}


    \item Sei $f: [0, \infty) \to (0, \infty)$ monoton fallend. Wie hängen $\sum_{k=1}^\infty f(k)$ und $\int_1^\infty f(t)\,dt$ zusammen?

    \textbf{Answer:} $\sum_{k=1}^\infty f(k)$ converges if and only if $\int_1^\infty f(t)\,dt$ converges. Furthermore  $$\sum_{k=2}^\infty f(k) \le \int_1^\infty f(t)\,dt \le \sum_{k=1}^\infty f(k)$$
    \item Welche der folgenden uneigentlichen Riemann-Integrale existieren? Welche konvergieren absolut?
    \begin{itemize}
        \item $\int_1^\infty \cos{t}\,dt$
        \item $\int_1^\infty \cos{t^2}\,dt$
        \item $\int_1^\infty \frac{\cos{t}}{t} \,dt$
        \item $\int_1^\infty \frac{\cos^2{t}}{t} \,dt$
        \item $\int_1^\infty \frac{\cos{t}}{t^2} \,dt$
        \item $\int_1^\infty \frac{\cos^2{t}}{t^2} \,dt$
    \end{itemize}

    \textbf{Answer: TODO}
    \item Warum konvergiert die Reihe der Riemannsche $\zeta$-Funktion $\zeta(s) := \sum_{k=1}^\infty \frac{1}{k^s}$ für $\Re{s} > 1$?

    \textbf{Answer:} Let $q = \Re{s}$. Then $\sum_{k=1}^\infty \abs{\frac{1}{k^s}} = \sum_{k=1}^\infty \frac{1}{k^q}$ converges iif $\int_1^\infty \frac{1}{t^q}\,dt$ converges. But this improper integral exists exactly when $1 < q \in \RR$. Thus if $\Re{s} > 1$ then $\sum_{k=1}^\infty \frac{1}{k^s}$ is absolutely convergent, and thus also converges conditionally.

    \item Wie ist die Gamma-Funktion definiert, und welche Funktionalgleichung erfüllt sie?

    \textbf{Answer:} Let $\Gamma(\alpha) := \int_0^\infty t^{\alpha -1} e^{-t}\,dt~(\forall \alpha \in \CC, \Re\alpha > 0)$. Then $\Gamma(\alpha + 1) = \alpha \Gamma(\alpha)$ and since $\Gamma(1) = 1 = 0!$ it holds follows from a simple induction that $\Gamma(n+1) = n!~(\forall n \in \NN_0)$.
    \item Wie lautet die Stirling-Formel zur Approximation von $n!$?

    \textbf{Answer:}
    \item Wie ist die Faltung $\varphi \star f$ von zwei Funktionen $\varphi, f: \RR \to \RR$ definiert?

    \textbf{Answer:}
    \item Was ist eine Dirac-Folge? Gib eine Definition und wenigstens ein Beispiel an.

    \textbf{Answer:}
    \item Sei $\varphi_n: \RR \to \RR$ eine Dirac-Folge und sei $f: \RR \to \RR$ stetig mit $f(x) = 0$ für $\abs{x} \ge 1$. Bestimme $\lim_{n\to\infty} (\varphi_n * f)$  für jedes $x \in \RR$.

    \textbf{Answer:}
    \item Wie lautet den Approximationssatz von Weierstrass?

    \textbf{Answer:}
    \item Welche Funktionen können durch Polynome gleichmäßig approximiert werden? Mit welcher Grundidee lassen sich approximierende Polynome zu einer gegebenen Funktion $f$ konstruieren?

    \textbf{Answer:}
    \item Was ist ein Hilbertraum? Gib zwei unendlich-dimensionale Beispiele.

    \textbf{Answer:}
    \item Was sind die Fourierkoeffizienten einer $2\pi$-periodischen Funktion $f: \RR \to \CC$ ? Wie lautet die Fourier-Reihe zu $f$ ?

    \textbf{Answer:}
    \item Beschreibe wie $2\pi$-periodische $f \in L^2$, die Fourier-Reihe $\hat{f}(t) := \sum_{k\in\ZZ}{c_k e^{ikt}}$, und Fourier-Koeffizienten $(c_k)_{k \in \ZZ} \in \ell^2$ miteinander zusammenhängen.

    \textbf{Answer:}
    \item Wie hängen Fourier-Reihen mit Obertönen in der Musik zusammen?

    \textbf{Answer:}

\end{enumerate}

\end{document}
