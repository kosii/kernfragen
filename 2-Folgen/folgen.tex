\documentclass[11pt]{article}

\usepackage{amsmath,amssymb,amsfonts}
\usepackage{graphicx}
\usepackage{mathtools}
\usepackage[dvipsnames]{xcolor}
\usepackage{}


\newcommand{\norm}[1]{\left\lVert#1\right\rVert}
\newcommand{\abs}[1]{\left|#1\right|}
\newcommand{\sumn}[4]{\sum_{#1=#2}^{#3}{#4}}
\newcommand{\RR}[0]{\mathbb{R}}
\newcommand{\CC}[0]{\mathbb{C}}
\newcommand{\QQ}[0]{\mathbb{Q}}
\newcommand{\ZZ}[0]{\mathbb{Z}}
\newcommand{\NN}[0]{\mathbb{N}}
\newcommand{\KK}[0]{\mathbb{K}}
\newcommand{\smallo}[0]{{\scriptstyle \mathcal{O}}}
\DeclarePairedDelimiter\floor{\lfloor}{\rfloor}
\newcommand{\slim}[2]{\lim_{#1\to\infty}{#2}}

\setlength{\topmargin}{-.5in} \setlength{\textheight}{9.25in}
\setlength{\oddsidemargin}{0in} \setlength{\textwidth}{6.8in}


\begin{document}

\noindent{\bf Kernfragen - Folgen\hfill Balázs Kossovics \hfill WS21-22 Analysis I.}


\medskip\hrule
\begin{enumerate}
    \item \label{label:limit} Wann nennt man eine Folge $x_n$ reeller Zahlen konvergent und wann uneigentlich konvergent?

    \textbf{Answer:} The $(x_n) \in \RR$ sequence converges to $l \in \RR$ if $\forall \epsilon >0~\exists N \in \NN\colon \forall n > N\colon \abs{x_n - l} < \epsilon$. If such $l$ exists, then it's unique.

    If there is no such $l \in \RR$ but $\forall X \in \RR~\exists N \in \NN\colon \forall n > N\colon x_n > X$ then we say that $(x_n)$ tends to infinity, and we denote it with $\lim_{x\to\infty} x_n = +\infty$. If $-x_n$ tends to $+\infty$, then we say that $x_n$ tends to negative infinity, and we denote it with $\lim_{n\to\infty}x_n = -\infty$

    \item Wie ist der Grenzwert einer Folge definiert?

    \textbf{Answer:} If a sequence converges, then its limit is the uniquely defined $l$ in Question $\ref{label:limit}$.


    \item \label{label:e} Was sind Supremum und Infimum der Folgen $(1 + \frac{1}{n})^n$ und $(1 + \frac{1}{n} )^{n+1}$ ?

    \textbf{Answer:}

    $x_n$ increasing:
    $$\begin{aligned}\frac{x_{n+1}}{x_n} &= \left(\frac{n+2}{n+1}\right)^{n+1} / \left(\frac{n+1}{n}\right)^n = \left(\frac{n+2}{n+1}\frac{n}{n+1}\right)^n\frac{n+2}{n+1} = \left(\frac{(n+1)^2 -1}{(n+2)^2}\right)^n\frac{n+2}{n+1} \\&= \left(1 - \frac{1}{(n+1)^2}\right)^n \frac{n+2}{n+1} \ge \left(1 -\frac{n}{(n+1)^2}\right)\frac{n+2}{n+1} = 1 + \frac{1}{(n+3)^3} > 1\end{aligned}$$

    Where the $\ge$ holds because of Bernoulli's inequality, and the $=$ after is just algebraic transformations.

    $y_n$ decreasing:

    $$\begin{aligned}
        \frac{y_n}{y_{n+1}} &= \left(\frac{n+1}{n}\right)^{n+1} / \left(\frac{n+2}{n+1}\right)^{n+2} = \left(\frac{n+1}{n}\frac{n+1}{n+2}\right)^{n+1} \frac{n+1}{n+2} = \left(\frac{(n+1)^2}{(n+1)^2 -1}\right)^{n+1} \frac{n+1}{n+2} \\
        &= \left(1+\frac{1}{(n+1)^2 -1}\right)^{n+1} \frac{n+1}{n+2} \ge \left(1 + \frac{n+1}{(n+1)^2 - 1}\right) \frac{n+1}{n+2} > \left(1 + \frac{n+1}{(n+1)^2}\right)\frac{n+1}{n+2} = 1
    \end{aligned}$$

    Where the $\ge$ holds because of Bernoulli's inequality and the $>$ holds because of incereasing the denominator, thus decreasing the value of the whole fraction.

    It holds furthermore that $x_n = \left(1 + \frac{1}{n}\right)y_n$, thus $x_n$ converges exactly when $y_n$ converges. Since they are both monoton sequences and $x_n < y_n$, they are both bounded, and consequently convergent, and they have the same limit, which we denote by $e$. Thus $\sup_{n \in \NN} x_n = e = \inf_{n \in \NN} y_n$ and $\inf_{n \in \NN} x_n = x_1 = 2$ and $\sup_{n \in \NN} y_n = y_1 = 4$

    \item Wie sind Häufungswerte einer Folge $x_n$ komplexer Zahlen definiert?

    \textbf{Answer:}
    Let $(x_n) \in \RR$ (or $(x_n) \in \CC$). We call $\zeta \in \RR$ (or $\zeta \in \CC$) a limit point of the $(x_n)$ sequence if exists $k_n\colon\NN \to \NN$ index sequence with $k_n \ge n$ such that $\lim_{n\to\infty}x_{k_n} = \zeta$

    \item Was sind die Häufungswerte der Folge $(-1)^n + \frac{1}{n}$?

    % Question 5
    \textbf{Answer:} Let $x_n = (-1)^n + \frac{1}{n}$, now $\limsup_{n\to\infty} = 1$ and $\liminf_{n\to\infty} = -1$, thus the sequence has at least two limit points. Since $\forall \epsilon > 0~\exists N \in \NN~\forall n > N\colon$ either $\abs{x_n - 1} < \epsilon$ or $\abs{x_n - (-1)} < \epsilon$, the sequence cannot have any other limit points.

    \item Was sind Limes superior und Limes inferior einer reellwertigen Folge? Wann existieren sie? Wann stimmen Limes superior und Limes inferior überein?

    \textbf{Answer:} Consider an arbitrary $(x_n) \in \RR$ sequence. The
    \begin{itemize}
        \item Limit superior: $\limsup_{n\to\infty} x_n = \lim_{n\to\infty} \sup_{k \ge n}  \{x_k\}$
        \item Limit inferior: $\liminf_{n\to\infty} x_n = \lim_{n\to\infty} \inf_{k \ge n}  \{x_k\}$
    \end{itemize}
    For any bounded sequence its $\limsup$ and $\liminf$ exist. The $\limsup$ and $\liminf$ of a sequence are equal if and only if the sequence converges, and in this case $$\lim_{n\to\infty} x_n = \limsup_{n\to\infty} x_n = \liminf_{n\to\infty} x_n$$

    \item Konstruiere eine reelle Folge, die jede reelle Zahl aus Häufungswert hat.

    \textbf{Answer:} Consider the following enumeration of the rationals: $0, \frac{1}{1}, \frac{1}{2}, \frac{2}{1}, \frac{1}{3}, \frac{2}{3}, \frac{3}{1}, \frac{1}{4}, \frac{2}{3}, \frac{3}{2}, \frac{4}{1}, \frac{1}{5}, \frac{2}{4}, \dots$
    and the $\phi\colon \NN^2 \to \NN$ bijection given by $\phi(p, q) = \frac{(p + q -1)(p + q - 2)}{2} + p$

    Define the following sequence: $x_0 = 0, x_1 = \frac{1}{1}$ and $\forall n > 1\colon x_n = \frac{p}{q}$ such that $\phi(p, q) = n$. The so defined sequence will correspond to the above enumeration and thus every non-negative rational will occur in $x_n$ eventually, and since $\QQ$ is dense in $\RR$, every non-negative real will be a limit point of $x_n$. From $x_n$ we can construct a new sequence with $x_0, x_1, -x_1, x_2, -x_2, x_3, -x_3, \dots$ that will have every real as a limit point.

    %Question 8
    \item Wann heißt eine Menge $A \subseteq B$ dicht in $B$? Gilt Dichtheit für $\ZZ \subseteq \QQ$, $\QQ \subseteq \RR$, $\RR \setminus \QQ \subseteq \RR$, $\QQ + i\QQ \subseteq \CC$, $\QQ + i(\RR\setminus\QQ) \subseteq \CC$? Warum?

    \textbf{Answer:} $A$ is dense in $B$ if $\forall b \in B~\exists (a_n)\colon \NN \to A\colon \lim_{n\to\infty} a_n = b$

    Or equivalently: $\forall b \in B~\forall \epsilon > 0~\exists a \in A\colon \abs{a -b} < \epsilon$

    Or equivalently: every point of $B$ is a limit point of $A$.
    \begin{itemize}
        \item $\ZZ \subseteq \QQ$: no, there is no sequence in $\ZZ$ that converges to $-1/12$
        \item $\QQ \subseteq \RR$: yes, $\forall r\in \RR~\exists(x_n)\in \QQ\colon \lim_{n\to\infty}x_n = r$, for example with $x_n = \frac{\floor{r n}}{n} \in \QQ$
        \item $\RR \setminus \QQ \subseteq \RR$: yes, $\forall r\in\RR~\exists(x_n) \in (\RR \setminus \QQ)\colon \lim_{n\to\infty} x_n = r$, for example with $x_n = \frac{\floor{n \sqrt{2} r}}{n \sqrt{2}} \in \RR \setminus \QQ$ (because a rational divided by an irrational is always irrational)
        \item $\QQ + i\QQ \subseteq \CC$ yes, because $\QQ$ is dense an $\RR$, thus $\forall a + ib \in \CC (a, b \in RR)~\exists (x_n), (y_n) \in \QQ\colon \lim_{n\to\infty} x_n = a, \lim_{n\to\infty} y_n = b$ and thus $\lim_{n\to\infty} x_n + i y_n = a + b i = z$ (explanation: a sequence $(z_n) = (a_n + i b_n) \in \CC~(a_n, b_n \in \RR)$ convergent if and only if $a_n$ and $b_n$ both converges, and in this case $\lim_{n\to\infty} z_n = \lim_{n\to\infty} a_n + i\lim_{n\to\infty}b_n$).
        \item $\QQ + i(\RR\setminus\QQ) \subseteq \CC$ yes, because $\QQ$ and $\RR \setminus \QQ$ are both dense in $\RR$, so the previous point hold, just with $(b_n) \in \RR\setminus\QQ$
    \end{itemize}
    \item Was besagt das Konvergenzkriterium von Cauchy? Warum gilt es?

    \textbf{Answer:} Consider an $(x_n) \in \RR$ sequence. $x_n$ is convergent if and only if $\forall \epsilon > 0~\exists N \in \NN\colon \forall n, m > N\colon \abs{x_n - x_m} < \epsilon$. Because $\RR$ is complete (the Cauchy criterum doesn't hold in $\QQ$ for example).

    \item Welche monotonen Folgen besitzen einen Grenzwert?

    \textbf{Answer:} Bounded monoton sequences.

    \item Welche dieser Folgen konvergieren für $n \to \infty$? Was sind ggf. ihre Grenzwerte, einschließlich $\pm \infty$?

    \hspace*{\fill}
    $\frac{n^2}{3n - 2}, \hfill \frac{3n^2 - 2}{2n^2+3},\hfill \frac{2^n}{n!} \hfill \sqrt[n]{n} \hfill q^{1/n}~(\text{für reelle }q > 0), \hfill \sqrt{n+1} - \sqrt{n}\hfill$
    \hspace*{\fill}

    \textbf{Answer:}
    \begin{enumerate}
        \item $\lim_{n\to\infty}\frac{n^2}{3n - 2} = \lim_{n\to\infty}\frac{n}{3 - 2/n} = +\infty$
        \item $\lim_{n\to\infty}\frac{3n^2 - 2}{2n^2+3} = \lim_{n\to\infty}\frac{3 - 2/n^2}{2+3/n^2} = 3/2$
        \item $\lim_{n\to\infty}\frac{2^n}{n!} = \lim_{n\to\infty}\prod_{k=1}^n{\frac{2}{i}} = \frac{4}{3} \lim_{n\to\infty}\prod_{k=4}^n{\frac{2}{k}} < \frac{4}{3} \lim_{n\to\infty}{\left(\frac{1}{2}\right)}^{n-3} = 0 $
        \item Consider first $q>1$. Let $h_n = q^{1/n} - 1 > 0$, thus $(1+h_n)^n = q~(\forall n \in \NN)$. From Bernoulli's inequality $q \ge 1 + n h_n>1~(\forall n\in\NN)$, from which it follows that $h_n \to 0~(n\to\infty)$. Consequently $q^{1/n} \to 1$. Now consider $0 < q < 1$: from the previous $(1/q)^{1/n} \to 1~(n \to \infty)$, thus from the algebraic properties of sequences we get $q^{1/n} \to 1$ as well.
        \item $\lim_{n\to\infty} \sqrt{n+1} - \sqrt{n} = \lim_{n\to\infty} \frac{( \sqrt{n+1} - \sqrt{n})( \sqrt{n+1} + \sqrt{n})}{ \sqrt{n+1} + \sqrt{n}} = \lim_{n\to\infty}\frac{n+1 - n}{\sqrt{n+1} + \sqrt{n}} \\= \slim{n} {\frac{1}{\sqrt{n+1} + \sqrt{n}}} = 0$
    \end{enumerate}

    %Question 12
    \item Wie lautet der Satz von Bolzano-Weierstraß?

    \textbf{Answer:} Every bounded sequence in $\RR$ or $\CC$ has a limit point. Conversely: every convergent sequence in $\RR$ or $\CC$ is bounded.

    %Question 13
    \item Wieviele Häufungswerte kann eine beschränkte, reellwertige Folge mindestens/höchstens besitzen?

    \textbf{Answer:} A convergent sequence has exactly one limit point (independently of being bounded or not). Any non convergent bounded sequence has at least 2 limit points: its limit superior and limit inferior. Consider now a sequence that enumerates all the rationals between $0$ and $1$. The rationals are countable, thus such sequence exists, and since $\QQ$ is dense in $\RR$, every real number in the $[0, 1]$ interval is a limit point of such a sequence

    \item Wie kann man den Kreisumfang durch Approximation mit regulären $2^n$-Ecken bestimmen?

    \textbf{Answer:}
    \item Gib zwei Definitionen der Eulerschen Zahl $e$. Warum stimmen beide Definitionen überein?

    \textbf{Answer:} Consider sequences $\hat{e}_n = \sumn{k}{0}{n}{\frac{1}{k!}}$ and $e_n = \left(1 + \frac{1}{n}\right)^n$.  From Question \ref{label:e} we already know that $e_n$ converges, let $e = \slim{n}{e_n}$. Furhermore from the Binomial theorem it's easy to see that $e_n \le \hat{e}_n~(\forall n\in\NN)$:

    $$e_n = \left(1+\frac{1}{n}\right)^n = \sumn{k}{0}{n}{\binom{n}{k}}\frac{1}{n^k} = \sumn{k}{0}{n}{\frac{1}{k!}\frac{n}{n} \frac{n-1}{n} \frac{n-2}{n} \dots \frac{n-k+1}{n}} < \sumn{k}{0}{n}{\frac{1}{k!}} = \hat{e}_n$$

    Since $\hat{e}_n$ is a sum of positive terms, it's clear that it increasing, and it's bounded:

    $$\hat{e}_n = \sumn{k}{0}{n}{\frac{1}{k!}} = 1 + 1 + \frac{1}{2} + \frac{1}{3!} + \frac{1}{4!} + \dots + \frac{1}{k!} < 2 + \frac{1}{2} + \frac{1}{2^2} + \frac{1}{2^3} + \dots < 3$$

    thus $\hat{e}_n$ also converges, and let $\hat{e} = \slim{n}{\hat{e}_n}$. Since $e_n < \hat{e}_n$, it follows that $e \le \hat{e}$. Furthermore $\forall m \le n$ holds that

    $$\begin{aligned}
        e_m &= \left(1+\frac{1}{m}\right)^m =\sumn{k}{0}{m}{\binom{m}{k}}\frac{1}{m^k} = \sumn{k}{0}{m}{\frac{1}{k!}\left(1 -\frac{1}{m}\right)\left(1 - \frac{2}{m}\right)\cdots\left(1 - \frac{k-1}{m}\right)} \\
        &\le \sumn{k}{0}{m}{\frac{1}{k!}\left(1 -\frac{1}{n}\right)\left(1 - \frac{2}{n}\right)\cdots\left(1 - \frac{k-1}{n}\right)} \\
        &< \sumn{k}{0}{n}{\frac{1}{k!}\left(1 -\frac{1}{n}\right)\left(1 - \frac{2}{n}\right)\cdots\left(1 - \frac{k-1}{n}\right)} = e_n
    \end{aligned}$$

    The important part here is

    $$\sumn{k}{0}{m}{\frac{1}{k!}\left(1 -\frac{1}{n}\right)\left(1 - \frac{2}{n}\right)\cdots\left(1 - \frac{k-1}{n}\right)} < e_n$$

    By taking first the $n\to\infty$ limit we get $\hat{e}_m \le e$, and the taking the $m\to\infty$ limit we get $\hat{e} \le e$. Together with the previous $e = \slim{n}{e_n} = \slim{n}{\hat{e}_n} = \hat{e}$ holds.
    \item Wie lauten die Abschätzungen für $n!$ von Stirling?

    \textbf{Answer:} $n^n/e^{n-1} \le n! \le n^{n+1}/e^{n-1}$ (proof: Lecture 13)
\end{enumerate}
\end{document}
