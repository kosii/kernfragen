\documentclass[11pt]{article}

\usepackage{amsmath,amssymb,amsfonts}
\usepackage{graphicx}
\usepackage[dvipsnames]{xcolor}
\usepackage{}


\newcommand{\norm}[1]{\left\lVert#1\right\rVert}
\newcommand{\abs}[1]{\left|#1\right|}
\newcommand{\sumn}[4]{\sum_{#1=#2}^{#3}{#4}}
\newcommand{\RR}[0]{\mathbb{R}}
\newcommand{\CC}[0]{\mathbb{C}}
\newcommand{\QQ}[0]{\mathbb{Q}}
\newcommand{\ZZ}[0]{\mathbb{Z}}
\newcommand{\NN}[0]{\mathbb{N}}
\newcommand{\KK}[0]{\mathbb{K}}
\newcommand{\smallo}[0]{{\scriptstyle \mathcal{O}}}

\setlength{\topmargin}{-.5in} \setlength{\textheight}{9.25in}
\setlength{\oddsidemargin}{0in} \setlength{\textwidth}{6.8in}


\begin{document}

\noindent{\bf Kernfragen - Differentiation\hfill Balázs Kossovics \hfill WS21-22 Analysis I.}


\medskip\hrule
\begin{enumerate}
    \item Wann heißt eine Funktion $f\colon \RR\to\RR$ in einem Punkt $x_0$ differenzierbar? Wie lässt sich die Ableitung geometisch interpretieren?
    
    \textbf{Answer:}
    If the $\lim_{x\to x_0}\frac{f(x_0 + h) - f(x_0)}{h}$ exists, then we call the function $f$ differentiable in the $x_0$ point. We call the value of $\lim_{x\to x_0}\frac{f(x_0 + h) - f(x_0)}{h}$ the derivative of the function $f$ in point $x_0$ and we denote it with $f^\prime(x_0)$. Whenever the derivative of a function exists, it's unique.

    The value of the $f^\prime(x_0)$ is the coefficient of $x$ in the best linear approximation of $f$ at point $x_0$, and it's the slope of the tangent line drawn to the function at the point $(x_0, f(x_0))$.

    \item  Gib Beispiele für Funtionen $f\colon \RR\to\RR$ an, die
    \begin{enumerate}
        \item stetig, aber in $x_0 = 0$ nicht differenzierbar;
        \item differenzierbar, aber nicht gleichmäßig stetig;
        \item differenzierbar, aber in $x_0 = 0$ nicht stetig differenzierbar sind
    \end{enumerate}

    \textbf{Answer:}

    \begin{enumerate}
        \item $f(x) = \abs{x}$
        \item $f(x) = x^2$
        \item $f(x) = \begin{cases}
            x^2&x\ge0\\
            -x^2&x<0
        \end{cases}$
    \end{enumerate}

    % Question 3
    \item Was bedeuten die Landau-Symbole $\smallo(h)$, $\mathcal{O}(h^2)$ und ${ \scriptstyle \mathcal{O}}(1)$? Wie lassen sich Stetigkeit und Differenzierbarkeit mit ihrer Hilfe ausdrücken?
    
    \textbf{Answer:}
    \begin{enumerate}
        \item $f(h) = \smallo(h) \Leftrightarrow \lim_{h\to0}\frac{f(h)}{h} = 0$
        \item $f(h) = \mathcal{O}(h^2) \Leftrightarrow \limsup_{h\to0}\abs{\frac{f(h)}{h^2}} < \infty$
        \item $f(h) = \smallo(1) \Leftrightarrow \lim_{h\to0} f(h) = 0$
    \end{enumerate}

    If there is a number $\alpha \in \RR$ such that $f(x_0+h) = f(x_0) + \alpha h + \smallo(h)$, then we say that the function $f$ is differentiable in the $x_0$ point.

    We say that $f$ is continuous in $x_0$ if $f(x_0 + h) = f(x_0) + \smallo(1)$

    % Question 4
    \item Für welche reellen $\alpha$ ist $\abs{x}^\alpha$ in $x = 0$ reell differenzierbar?
    
    \textbf{Answer:}
    % Question 5
    \item Wie lautet die Produktregel für Ableitungen? Warum gilt sie (Beweis)?
    
    \textbf{Answer:}

    Consider two functions $f$ and $g$ that are both differentiable in some $x_0$ point of their domain. Then the $f g$ function is also differentiable in $x_0$ and $(f g)^\prime(x_0) = f^\prime(x_0) g(x_0) + f(x_0) g^\prime(x_0)$
    
    \textit{Proof:}
    $$\begin{aligned}
        &\lim_{h\to0}\frac{f(x_0 + h)g(x_0 + h) - f(x_0)g(x_0)}{h} \\
        =&\lim_{h\to0}\frac{f(x_0 + h)g(x_0 + h) - f(x)g(x_0 + h) + f(x)g(x_0 + h)- f(x_0)g(x_0)}{h}\\
        =&\lim_{h\to0}\frac{(f(x_0 + h)- f(x))g(x_0 + h) + f(x)(g(x_0 + h)- g(x_0))}{h}
    \end{aligned}$$

    Since $g$ is continuous in $x_0$ and the $\lim_{h\to0}\frac{f(x_0 + h)- f(x)}{h}$ and $\lim_{h\to0}\frac{g(x_0 + h)- g(x)}{h}$ exist, thus the above limit also exists and 
    $$\begin{aligned}&= \lim_{h\to0}\frac{f(x_0 + h)- f(x)}{h} g(x_0 + h) + f(x)\frac{g(x_0 + h)- g(x)}{h} \\&=f^\prime(x_0) g(x_0) + f(x_0) g^\prime(x_0)\end{aligned}$$

    \item Wie lauten Quotienten- und Kettenregel für Ableitungen?
    
    \textbf{Answer:}

    \textit{Division:} Suppose that both $f$ and $g$ functions are differentiable in $x_0$ and furthermore suppose that $g(x_0) \neq 0$. Then the function $\frac{f}{g}$ is also differentiable in $x_0$ and $\left(\frac{f}{g}\right)^\prime(x_0) = \frac{f^\prime(x_0)g(x_0)-f(x_0)g^\prime(x_0)}{g^2(x_0)}$
    
    \textit{Chain rule:} Suppose that $g$ is differentiable in some $x_0$ point and $f$ is differentiable in $y_0 = f(x_0)$. Then $f \circ g$ is also differentiable in $x_0$ and $(f \circ g)^\prime(x_0) = f^\prime(g(x_0))g^\prime(x_0)$

    \item Was sind die Ableitungen folgender Funktionen nach $x$?
    
    \hspace*{\fill}
    $e^x \sin{x} \hfill \frac{\sin{x}}{\cos{x}}\hfill \exp(-x^2)\hfill \log{\frac{1+x}{1-x}}\hfill x^x$
    \hspace*{\fill}

    \textbf{Answer:}
    \begin{enumerate}
        \item $(e^x \sin{x})^\prime = e^x \sin{x} + e^x \cos{x}$ from the product rule because $\exp^\prime = \exp$ and $\sin^\prime = \cos$
        \item Suppose that $x \neq k\pi~(k \in \ZZ)$. Then $\frac{\sin{x}}{\cos{x}} = \frac{\cos^2{x} + \sin^2{x}}{\cos^2{x}} = \frac{1}{\cos^2{x}}$ from the rule of division
        \item $(\exp(-x^2))^\prime = -2x\exp(-x^2)$ from the chain rule
        \item For $x > 1\colon (\log\frac{1+x}{1-x})^\prime = \frac{1}{\frac{1+x}{1-x}} (\frac{1+x}{1-x})^\prime = \frac{1-x}{1+x} \frac{(1-x) + (1+x)}{(1-x)^2}=\frac{2}{1-x^2}$
        \item For $x > 0\colon (x^x)^\prime = (\exp(x \log{x}))^\prime = x^x (\log{x} + x\frac{1}{x}) = x^x (\log{x} + 1)$
    \end{enumerate}

    %Question 8
    \item Wann besitzt eine Funktion $f\colon \RR\to\RR$ eine differenzierbare Umkehrfunktion $f^{-1}$?
    
    \textbf{Answer: (VERIFY)} Suppose that $f$ is differentiable everywhere and either strict monotone increasing or decreasing on $\RR$. Then $f$ is continuous (from the differentiability) and its inverse exists (because it's continuous and strict monotone increasing).

    The inverse function $f^{-1}$ will be differentiable in some $y_0 = f(x_0)$ point if $f^\prime(x_0) \neq 0$ and in this case $(f^{-1})^\prime(y_0) = \frac{1}{f^\prime(f^{-1}(y_0))}$
    \item Wie lautet der Mittelwertsatz (der Differentialrechnung)? Wie lautet der Satz von
    Rolle?

    \textbf{Answer:}

    \textit{Mean Value Theorem:} Consider some continuous function $f\colon [a, b] \to \RR$ which is differentiable on $(a, b)$. Then $\exists c \in (a, b)\colon \frac{f(b) - f(a)}{b - a} = f^\prime(c)$

    \textit{Rolle:} Consider some continuous function $f\colon [a, b] \to \RR$ which is differentiable on $(a, b)$, and suppose that $f(a) = f(b)$. Then $\exists c\in(a, b)\colon f^\prime(c) = 0$

    \item Warum gilt der Satz von Rolle (Beweisskizze)?
    
    \textbf{Answer: TODO} 

    \item Wie lauten die Regeln von de l'Hôpital?
    
    \textbf{Answer: CHECK} Consider two functions $f, g$ that are differentiable in some $x_0$ point. If $f(x_0) = g(x_0) = 0$ and $\lim_{x \to x_0}\frac{f^\prime(x)}{g^\prime(x)}$ exists, then $\lim_{x \to x_0}\frac{f(x)}{g^(x)}$ exists as well and $\lim_{x \to x_0}\frac{f(x)}{g^(x)} = \lim_{x \to x_0}\frac{f^\prime(x)}{g^\prime(x)}$
    \item Welche Werte haben die stetigen Fortsetzungen folgender Funktionen in $x = 0$?
    
    \hspace*{\fill}
    $\frac{\sin{x}}{x} \hfill \frac{\cos{x} - 1}{x^2} \hfill \frac{\log(1+x)}{x} \hfill \frac{x}{e^x -1}$
    \hspace*{\fill}

    \textbf{Answer:}

    % Question 13
    \item Berechne
    $$\lim_{x \to +\infty}\frac{e^x - e^{-x}}{e^x + e^{-x}}$$

    \textbf{Answer:}
    % Question 14
    \item Wie lauten die Ungleichungen von Young und Hölder?
    
    \textbf{Answer:}
    \item Skizziere die Funktionen $\sin{x}$ und $\cos{x}$, beschreibe ihre Nullstellen, Ableitungen, Monotonie, Konvexität und Konkavität, und erläutere unsere Definition von $\pi$.

    \textbf{Answer:}
    \item Sei $f\colon \RR\to\RR$ eine zweimal differenzierbare Funktion. Welche (notwendige) Bedingung ist erfüllt, wenn $f$ an der Stelle $x_0$ ein lokales Maximum besitzt? Unter welcher (hinreichenden) Bedingung besitzt $f$ an der Stelle $x_0$ ein lokales Maximum?
    
    \textbf{Answer:}
    % Question 17
    \item Wann heißt eine Funktion $f\colon (a, b)\to\RR$ konvex? Wann heißt sie strikt konvex?
    
    \textbf{Answer:}
    \item Die Funktion $f\colon (a, b)\to\RR$ sei zweimal diferenzierbar. Wie lassen sich Konvexität
    und strikte Konvexität durch Bedingungen an die zweite Ableitung ausdrücken?

    \textbf{Answer:}
    \item Wieviele Minima bzw. Maxima kann eine strikt konvexe Funktion $f\colon [a,b]\to\RR$ haben? (Gib alle möglichen Zahlen an.)
    
    \textbf{Answer:}
    \item Wo sind (reelle) Potenzreihen differenzierbar? Wie lautet die Ableitung?
    
    \textbf{Answer:}
    \item Wie ist der Raum $\mathcal{BC}^1(\RR,\RR)$ definiert? Was bedeutet seine Vollständigkeit für die Vertauschbarkeit von Differentiation und Grenzwertbildung einer Funktionenfolge $f_n\colon \RR\to\RR$?
    
    \textbf{Answer:}
    \item Wie lautet das $n$-te Taylor-Polynom? Wie kann das Restglied ausgedrückt werden?
    
    \textbf{Answer:}
    \item Wann (und wo) wird eine reelle Funktion durch ihre Taylor-Reihe dargestellt? Gib ein Beispiel und ein Gegenbeispiel.
    
    \textbf{Answer:}
    \item Wie lauten die Taylor-Reihen folgender Funktionen in $x_0 = 0$?
    \begin{center}
        $e^x$, $\sin{x}$, $\arctan{x}$, $(1+x)^\alpha$, $\log(1+x)$
    \end{center}

    \textbf{Answer:}
\end{enumerate}

\end{document} 
