\documentclass[11pt]{article}

\usepackage{amsmath,amssymb,amsfonts}
\usepackage{graphicx}
\usepackage{mathtools}
\usepackage{bbm}
\usepackage[dvipsnames]{xcolor}
\usepackage{}


\newcommand{\norm}[1]{\left\lVert#1\right\rVert}
\newcommand{\abs}[1]{\left|#1\right|}
\newcommand{\sumn}[4]{\sum_{#1=#2}^{#3}{#4}}
\newcommand{\RR}[0]{\mathbb{R}}
\newcommand{\CC}[0]{\mathbb{C}}
\newcommand{\QQ}[0]{\mathbb{Q}}
\newcommand{\ZZ}[0]{\mathbb{Z}}
\newcommand{\NN}[0]{\mathbb{N}}
\newcommand{\KK}[0]{\mathbb{K}}
\newcommand{\smallo}[0]{{\scriptstyle \mathcal{O}}}
\DeclarePairedDelimiter\floor{\lfloor}{\rfloor}
\newcommand{\slim}[2]{\lim_{#1\to\infty}{#2}}
\renewcommand{\Re}[0]{\operatorname{Re}}
\renewcommand{\Im}[0]{\operatorname{Im}}
\DeclareMathOperator{\Leb}{Leb}

\setlength{\topmargin}{-.5in} \setlength{\textheight}{9.25in}
\setlength{\oddsidemargin}{0in} \setlength{\textwidth}{6.8in}


\begin{document}

\noindent{\bf Kernfragen - Metrische Räume\hfill Balázs Kossovics \hfill 2022 SS - Analysis II.}


\medskip\hrule
\begin{enumerate}
    %1
    \item Was ist eine Metrik? Gib drei verschiedene Beispiele!

    \textbf{Answer:} Consider $E$ arbitrary set. The function $d: E \times E \to \RR$ is a metric, if satisfies the following conditions:
    \begin{itemize}
        \item $\forall x, y \in E\colon d(x, y) = 0 \Leftrightarrow x = y$
        \item $\forall x, y \in E\colon d(x, y) = d(y, x)$
        \item $\forall x, y, z \in E\colon d(x, z) \le d(x, y) + d(y, z)$
    \end{itemize}
    \item Wie sind offene Teilmengen eines metrischen Raumes definiert, wie abgeschlossene Teilmengen?

    \textbf{Answer:} Consider $(E, d)$ metric space, and let $x \in E$. $\forall \varepsilon > 0$ let's define $B_\varepsilon(x) := \left\{ y \in E \bigm | d(x, y) < \varepsilon\right\}$, the open ball of radius $\varepsilon$ centered around $x$. The $A \subset E$ set is open, if $\forall x \in A\colon \exists \varepsilon > 0\colon B_\varepsilon(x) \subset A$. The $B \subset E$ set is closed, if its complement $E \setminus B$ is open.
    \item Sind $\left\{\begin{tabular}{c}
        endliche\\
        abzählbare\\
        beliebige\\
    \end{tabular}\right\}$ $\left\{\begin{tabular}{c}
    Durchschnitte\\
    Vereinigungen
    \end{tabular}\right\}$ $\left\{\begin{tabular}{c}
        offener\\
        abgeschlossener
    \end{tabular}\right\}$ Mengen wieder offen bzw. abgeschlossen?

    \textbf{Answer:} Arbitrary union of open sets is open. Finite intersection of open sets is open. Arbitrary intersection of open sets is in general not open: consider $\RR$ with the standard metric, then $\cap_{n \in \NN} (-\frac{1}{n}, \frac{1}{n}) = \left\{0\right\}$ closed.

    Arbitrary intersection of closed sets is closed. Finite union of closed sets is closed. Arbitrary union of closed sets is in general not closed: consider $\RR$ with the standard metric, then $\cup_{n \in \NN}\left[\frac{1}{n}, 1-\frac{1}{n}\right] = (0, 1)$ open.

    \item Wie sind Abschluss, Inneres und Rand einer Teilmenge eines metrischen Raumes definiert?

    \textbf{Answer:} Consider $(E, d)$ metric space, and let $A \subset E$.

    The closure of $A$ is $\overline{A} = \left\{x \in E \bigm| \forall \varepsilon > 0 \colon B_\varepsilon(x) \cap A \neq \emptyset \right\}$

    The inner of $A$ is $\mathring{A} = \left\{x \in A \bigm| \exists \varepsilon > 0\colon B_\varepsilon(x) \subset A \right\}$

    The boundary of $A$ is $\partial A = \overline{A} \setminus \mathring{A}$
    \item Was sind Abschluss, Inneres und Rand folgender Teilmengen der reellen Zahlen (mit deren Standardmetrik)?


    \hspace*{\fill}
    $\ZZ \hfill \QQ \hfill \cup_{k\in\NN}\left(\frac{1}{k+1}, \frac{1}{k}\right)$
    \hspace*{\fill}

    \textbf{Answer:}
    \begin{itemize}
        \item $\overline{\ZZ} = \ZZ$ because $\ZZ$ is only consist of isolated points. $\mathring{\ZZ} = \emptyset$, since $\mathring{\ZZ} = \RR \setminus \overline{\RR \setminus \ZZ} = \RR \setminus \RR = \emptyset$. $\partial \ZZ = \overline{\ZZ} \setminus \mathring{\ZZ} = \ZZ$
        \item $\QQ$ is dense in $\RR$, thus $\overline{\QQ} = \RR$. $\mathring{\QQ} = \RR \setminus \overline{\RR \setminus \QQ} = \RR \setminus \RR = \emptyset$, since the irrationals are also dense in $\RR$. $\partial\QQ = \overline{\QQ} \setminus \mathring{\QQ} = \RR$
        \item $A := \cup_{k\in\NN}\left(\frac{1}{k+1}, \frac{1}{k}\right) = (0, 1) \setminus \left\{\frac{1}{n} \bigm| n \in \NN\right\}$. $\overline{A} = [0, 1]$, $\mathring{A} = A$ since it's an union of open sets, thus it's also open and so the interior is itself. $\partial{A} = \overline{A} \setminus \mathring{A} = \left\{\frac{1}{n} \bigm| n \in \NN\right\} \cup \left\{0\right\}$
    \end{itemize}

    \item Wann ist eine Teilmenge eines metrischen Raumes dicht?

    \textbf{Answer:} Consider $(E, d)$ metric space and $A \subset E$. $A$ is dense in $E$, if $\overline{A} = E$.

    \item Wann ist ein metrischer Raum zusammenhängend?

    \textbf{Answer:} Consider $(E, d)$ metric space. It's connected, if $\forall A, B \subset E\colon A, B$ open, $A \cup B = E, A \cap B = \emptyset \Rightarrow$ either $A = \emptyset$ or $B = \emptyset$
    \item Wie sind Zusammenhangskomponenten eines metrischen Raumes definiert? Wann heißt ein metrischer Raum total unzusammenhängend?

    \textbf{Answer:} Consider $(E, d)$ metric space. For some $x\in E$ we define the connected component of the point $x$ as the union of all $C\subset E$ connected sets, that contain $x$. We denote the connected component of $x$ with $\mathcal{C}(x)$. $(E, d)$ is totally disconnected, if $\forall x\in E\colon \mathcal{C}(x) = \left\{x\right\}$

    \item Sind $\left\{\begin{tabular}{c}
    \text{endliche}\\
    \text{abzählbare}\\
    \text{beliebige}
    \end{tabular}\right\}$ $\left\{\begin{tabular}{c}
    \text{Durchschnitte}\\
    \text{Vereinigungen}
    \end{tabular}\right\}$ zusammenhängender Mengen wieder zusammenhängend? Gib gegebenfalls ein Gegenbeispiel!

    \textbf{Answer:} Union of connected sets is in general not connected: consider $\RR$ with the standard metric, and $[0, 1]$, $[2, 3]$ connected sets. $[0, 1] \cup [2, 3]$ is disconnected. Arbitrary union of connected sets $U_i$ is connected, as long as $\cap_i U_i \neq \emptyset$.

    Intersection of connected sets is not necessarily connected (think in $\RR^2$ about a $C$ and a $I$ shaped open set, intersecting each other only at the ends of the $C$, creating two disconnected sets).

    \item Sind $\left\{\begin{tabular}{c}
        \text{endliche}\\
        \text{abzählbare}\\
        \text{beliebige}
    \end{tabular}\right\}$ $\left\{\begin{tabular}{c}
        \text{Durchschnitte}\\
        \text{Vereinigungen}
    \end{tabular}\right\}$ kompakter Mengen wieder kompakt? Gib gegebenfalls ein Gegenbeispiel!

    \textbf{Answer: TODO}

    \item Warum ist in einem metrischen Raum jede konvergente Folge eine Cauchy-Folge? (Beweise!)

    \textbf{Answer:} Consider $(E, d)$ metric space, and an $(x_n) \in E$ convergent sequence. Let $\lim_{n \to \infty} x_n = x \in E$. Since $x_n$ converges, $\forall \varepsilon > 0\colon \exists N \in \NN\colon \forall n > N\colon d(x_n, x) < \frac{\varepsilon}{2}$. Now $\forall n, m > N\colon d(x_n, x_m) \le d(x_n, x) + d(x, x_m) = d(x_n, x) + d(x_m, x) < \frac{\varepsilon}{2} + \frac{\varepsilon}{2} < \varepsilon$, thus $(x_n)$ is a Cauchy sequence.

    \item Wann heißt ein metrischer Raum vollständig?

    \textbf{Answer:} Consider $(E, d)$ metric space. $E$ is complete, if every Cauchy sequence converges.

    \item Was sind generische Mengen?

    \textbf{Answer: TODO}
    \item Wie lautet der Satz von Baire?

    \textbf{Answer:} Consider $(E, d)$ complete metric space and countable many $U_i \subset E~(i \in \mathbb{N})$ open and dense sets. Then $\cap_{i \in \NN} U_i$ is dense in $E$.

    \item Wann heißt ein metrischer Raum perfekt?

    \textbf{Answer:} $(E, d)$ is perfect, if it doesn't contain isolated points. Or equivalently: $\forall x \in E\colon x \in \overline{E \setminus \left\{x\right\}}$

    \item Zeige, dass jeder nichtleere, vollständige, perfekte metrische Raum überabzählbar ist.

    \textbf{Answer: TODO}
    \item Was versteht man unter einer Cantor-Menge? Gib ein Beispiel an!

    \textbf{Answer:} Let $\emptyset \neq C \subset [0, 1]$. We call $C$ a Cantor set, if $C$ is complete, totally disconnected and perfect.

    \item Wann heißt eine Folge in einem metrischen Raum konvergent? Wann heißt sie Cauchy-Folge?

    \textbf{Answer:} Consider $(E, d)$ metric space, and $(x_n) \in E$ sequence. The $(x_n)$ sequence is convergent, if $\exists x\in E\colon \forall \varepsilon > 0\exists N \in \NN\colon \forall n > N\colon d(x, d_n) < \varepsilon$. We call $x$ the limit of the $(x_n)$ sequence and we say that $(x_n)$ converges agains $x$.

    The $(x_n) \in E$ sequence is a Cauchy sequence, if $\forall \varepsilon > 0\colon \exists N \in \NN\colon \forall m, n > N\colon d(x_n, x_m) < \varepsilon$.

    \item Wann heißt eine Abbildung zwischen metrischen Räumen stetig? Gib vier verschiedene (aber natürlich äquivalente) Definitionen!

    \textbf{Answer:} Consider $(E_1, d_1)$ and $(E_2, d_2)$ metric spaces. The $f\colon E_1 \to E_2$ function is continuous in $a \in E_1$ if $\forall \varepsilon > 0\colon \exists \delta > 0\colon \forall x \in E_1\colon d_1(a, x) < \delta\colon d_2(f(a), f(x)) < \varepsilon$.

    The $f\colon E_1 \to E_2$ is continuous if and only if one of the following equivalent characterisation holds (and if one holds, then all the other hold as well):
    \begin{enumerate}
        \item $f$ is continuous in $\forall a \in E_1$
        \item $\forall A \in E\colon f(\overline{A}) \subset \overline{f(A)}$
        \item $\forall A \subset E_2$ closed: $f^{-1}(A)$ is closed
        \item $\forall A \subset E_2$ open: $f^{-1}(A)$ is open
    \end{enumerate}

    \item Sind unter einer stetigen Abbildung $f: E \to E^\prime$ zwischen metrischen Räumen die $\left\{\begin{tabular}{c}
        \text{Bilder}\\
        \text{Urbilder}
    \end{tabular}\right\}$ $\left\{\begin{tabular}{c}
    \text{offener}\\
    \text{abgeschlossener}\\
    \text{vollstängiger}\\
    \text{zusammenhängender}\\
    \text{kompakter}
    \end{tabular}\right\}$ Teilmengen wieder $\left\{\begin{tabular}{c}
        \text{offener}\\
        \text{abgeschlossener}\\
        \text{vollstängiger}\\
        \text{zusammenhängender}\\
        \text{kompakter}
        \end{tabular}\right\}$?
    Gib gegebenfalls ein Gegenbeispiel!

    \textbf{Answer: TODO}
    \item Wann heißen zwei Metriken äquivalent?

    \textbf{Answer:}
    \item Wann heißen zwei Normen äquivalent?

    \textbf{Answer:}
    \item Sei $E$ ein metrischer Raum. Formuliere und beweise den Zwischenwertsatz für stetige Abbildungen $f\colon E \to \RR$

    \textbf{Answer:}
    \item Gib drei verschiedene (aber natürlich äquivalente) Definitionen von kompakten metrischen Räumen! Gib außerdem je ein Beispiel eines kompakten und eines nicht kompakten Raumes!

    \textbf{Answer:}
    \item Sei E ein metrischer Raum. Wann heißt eine Teilmenge $A \subset E$ relativ kompakt?

    \textbf{Answer:}
    \item Wie lassen sich die kompakten Teilmengen des $\RR^n$
    charakterisieren?

    \textbf{Answer:}
    \item Wie lautet der Satz von Arzela-Ascoli?

    \textbf{Answer:}
    \item Wann sagt man, dass eine Abbildung Lipschitz-stetig ist? Wann ist eine Lipschitzstetige Abbildung eine Kontraktion?

    \textbf{Answer:}
    \item Wie lautet der Banachsche Fixpunktsatz?

    \textbf{Answer:}
    \item Sei $E$ ein metrischer Raum und $f \colon E \to E$ stetig. Wann sagt man, dass $\emptyset \neq A \subset E$ selbstähnlich ist? Was ist ein \textit{iterated function system}?

    \textbf{Answer:}
\end{enumerate}

\end{document}
