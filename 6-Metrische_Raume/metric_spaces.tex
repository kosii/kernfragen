\documentclass[11pt]{article}

\usepackage{amsmath,amssymb,amsfonts}
\usepackage{graphicx}
\usepackage{mathtools}
\usepackage{bbm}
\usepackage[dvipsnames]{xcolor}
\usepackage{}


\newcommand{\norm}[1]{\left\lVert#1\right\rVert}
\newcommand{\abs}[1]{\left|#1\right|}
\newcommand{\sumn}[4]{\sum_{#1=#2}^{#3}{#4}}
\newcommand{\RR}[0]{\mathbb{R}}
\newcommand{\CC}[0]{\mathbb{C}}
\newcommand{\QQ}[0]{\mathbb{Q}}
\newcommand{\ZZ}[0]{\mathbb{Z}}
\newcommand{\NN}[0]{\mathbb{N}}
\newcommand{\KK}[0]{\mathbb{K}}
\newcommand{\smallo}[0]{{\scriptstyle \mathcal{O}}}
\DeclarePairedDelimiter\floor{\lfloor}{\rfloor}
\newcommand{\slim}[2]{\lim_{#1\to\infty}{#2}}
\renewcommand{\Re}[0]{\operatorname{Re}}
\renewcommand{\Im}[0]{\operatorname{Im}}
\DeclareMathOperator{\Leb}{Leb}

\setlength{\topmargin}{-.5in} \setlength{\textheight}{9.25in}
\setlength{\oddsidemargin}{0in} \setlength{\textwidth}{6.8in}


\begin{document}

\noindent{\bf Kernfragen - Metrische Räume\hfill Balázs Kossovics \hfill 2022 SS - Analysis II.}


\medskip\hrule
\begin{enumerate}
    %1
    \item Was ist eine Metrik? Gib drei verschiedene Beispiele!

    \textbf{Answer:} Consider $E$ arbitrary set. The function $d: E \times E \to \RR$ is a metric, if satisfies the following conditions:
    \begin{itemize}
        \item $\forall x, y \in E\colon d(x, y) \ge 0$ and $d(x, y) = 0 \Leftrightarrow x = y$
        \item $\forall x, y \in E\colon d(x, y) = d(y, x)$
        \item $\forall x, y, z \in E\colon d(x, z) \le d(x, y) + d(y, z)$
    \end{itemize}
    \item Wie sind offene Teilmengen eines metrischen Raumes definiert, wie abgeschlossene Teilmengen?

    \textbf{Answer:} Consider $(E, d)$ metric space, and let $x \in E$. $\forall \varepsilon > 0$ let's define $B_\varepsilon(x) := \left\{ y \in E \bigm | d(x, y) < \varepsilon\right\}$, the open ball of radius $\varepsilon$ centered around $x$. The $A \subset E$ set is open, if $\forall x \in A\colon \exists \varepsilon > 0\colon B_\varepsilon(x) \subset A$. The $B \subset E$ set is closed, if its complement $E \setminus B$ is open.
    \item Sind $\left\{\begin{tabular}{c}
        endliche\\
        abzählbare\\
        beliebige\\
    \end{tabular}\right\}$ $\left\{\begin{tabular}{c}
    Durchschnitte\\
    Vereinigungen
    \end{tabular}\right\}$ $\left\{\begin{tabular}{c}
        offener\\
        abgeschlossener
    \end{tabular}\right\}$ Mengen wieder offen bzw. abgeschlossen?

    \textbf{Answer:} Arbitrary union of open sets is open. Finite intersection of open sets is open. Arbitrary intersection of closed sets is closed. Finite union of closed sets is closed.

    \item Wie sind Abschluss, Inneres und Rand einer Teilmenge eines metrischen Raumes definiert?

    \textbf{Answer:} Consider $(E, d)$ metric space, and let $A \subset E$.

    The closure of $A$ is $\overline{A} = \left\{x \in E \bigm| \forall \varepsilon > 0 \colon B_\varepsilon(x) \cap A \neq \emptyset \right\}$

    The inner of $A$ is $\mathring{A} = \left\{x \in A \bigm| \exists \varepsilon > 0\colon B_\varepsilon(x) \subset A \right\}$

    The boundary of $A$ is $\partial A = \overline{A} \setminus \mathring{A}$
    \item Was sind Abschluss, Inneres und Rand folgender Teilmengen der reellen Zahlen (mit deren Standardmetrik)?


    \hspace*{\fill}
    $\ZZ \hfill \QQ \hfill \cup_{k\in\NN}\left(\frac{1}{k+1}, \frac{1}{k}\right)$
    \hspace*{\fill}

    \textbf{Answer:}
    \item Wann ist eine Teilmenge eines metrischen Raumes dicht?

    \textbf{Answer:}
    \item Wann ist ein metrischer Raum zusammenhängend?

    \textbf{Answer:}
    \item Wie sind Zusammenhangskomponenten eines metrischen Raumes definiert? Wann heißt ein metrischer Raum total unzusammenhängend?

    \textbf{Answer:}
    \item Sind $\left\{\begin{tabular}{c}
    \text{endliche}\\
    \text{abzählbare}\\
    \text{beliebige}
    \end{tabular}\right\}$ $\left\{\begin{tabular}{c}
    \text{Durchschnitte}\\
    \text{Vereinigungen}
    \end{tabular}\right\}$ zusammenhängender Mengen wieder zusammenhängend? Gib gegebenfalls ein Gegenbeispiel!

    \textbf{Answer:}
    \item Sind $\left\{\begin{tabular}{c}
        \text{endliche}\\
        \text{abzählbare}\\
        \text{beliebige}
    \end{tabular}\right\}$ $\left\{\begin{tabular}{c}
        \text{Durchschnitte}\\
        \text{Vereinigungen}
    \end{tabular}\right\}$ kompakter Mengen wieder kompakt? Gib gegebenfalls ein Gegenbeispiel!

    \textbf{Answer:}
    \item Warum ist in einem metrischen Raum jede konvergente Folge eine Cauchy-Folge?(Beweise!)

    \textbf{Answer:}
    \item Wann heißt ein metrischer Raum vollständig?

    \textbf{Answer:}
    \item Was sind generische Mengen?

    \textbf{Answer:}
    \item Wie lautet der Satz von Baire?

    \textbf{Answer:}
    \item Wann heißt ein metrischer Raum perfekt?

    \textbf{Answer:}
    \item Zeige, dass jeder nichtleere, vollständige, perfekte metrische Raum überabzählbar ist.

    \textbf{Answer:}
    \item Was versteht man unter einer Cantor-Menge? Gib ein Beispiel an!

    \textbf{Answer:}
    \item Wann heißt eine Folge in einem metrischen Raum konvergent? Wann heißt sie Cauchy-Folge?

    \textbf{Answer:}
    \item Wann heißt eine Abbildung zwischen metrischen Räumen stetig? Gib vier verschiedene (aber natürlich äquivalente) Definitionen!

    \textbf{Answer:}
    \item Sind unter einer stetigen Abbildung $f: E \to E^\prime$ zwischen metrischen Räumen die $\left\{\begin{tabular}{c}
        \text{Bilder}\\
        \text{Urbilder}
    \end{tabular}\right\}$ $\left\{\begin{tabular}{c}
    \text{offener}\\
    \text{abgeschlossener}\\
    \text{vollstängiger}\\
    \text{zusammenhängender}\\
    \text{kompakter}
    \end{tabular}\right\}$ Teilmengen wieder $\left\{\begin{tabular}{c}
        \text{offener}\\
        \text{abgeschlossener}\\
        \text{vollstängiger}\\
        \text{zusammenhängender}\\
        \text{kompakter}
        \end{tabular}\right\}$?
    Gib gegebenfalls ein Gegenbeispiel!

    \textbf{Answer:}
    \item Wann heißen zwei Metriken äquivalent?

    \textbf{Answer:}
    \item Wann heißen zwei Normen äquivalent?

    \textbf{Answer:}
    \item Sei $E$ ein metrischer Raum. Formuliere und beweise den Zwischenwertsatz für stetige Abbildungen $f\colon E \to \RR$

    \textbf{Answer:}
    \item Gib drei verschiedene (aber natürlich äquivalente) Definitionen von kompakten metrischen Räumen! Gib außerdem je ein Beispiel eines kompakten und eines nicht kompakten Raumes!

    \textbf{Answer:}
    \item Sei E ein metrischer Raum. Wann heißt eine Teilmenge $A \subset E$ relativ kompakt?

    \textbf{Answer:}
    \item Wie lassen sich die kompakten Teilmengen des $\RR^n$
    charakterisieren?

    \textbf{Answer:}
    \item Wie lautet der Satz von Arzela-Ascoli?

    \textbf{Answer:}
    \item Wann sagt man, dass eine Abbildung Lipschitz-stetig ist? Wann ist eine Lipschitzstetige Abbildung eine Kontraktion?

    \textbf{Answer:}
    \item Wie lautet der Banachsche Fixpunktsatz?

    \textbf{Answer:}
    \item Sei $E$ ein metrischer Raum und $f \colon E \to E$ stetig. Wann sagt man, dass $\emptyset \neq A \subset E$ selbstähnlich ist? Was ist ein \textit{iterated function system}?

    \textbf{Answer:}
\end{enumerate}

\end{document}
