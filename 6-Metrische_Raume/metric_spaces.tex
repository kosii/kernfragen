\documentclass[11pt]{article}

\usepackage{amsmath,amssymb,amsfonts}
\usepackage{graphicx}
\usepackage{mathtools}
\usepackage{bbm}
\usepackage[dvipsnames]{xcolor}
\usepackage{}


\newcommand{\norm}[1]{\left\lVert#1\right\rVert}
\newcommand{\abs}[1]{\left|#1\right|}
\newcommand{\sumn}[4]{\sum_{#1=#2}^{#3}{#4}}
\newcommand{\RR}[0]{\mathbb{R}}
\newcommand{\CC}[0]{\mathbb{C}}
\newcommand{\QQ}[0]{\mathbb{Q}}
\newcommand{\ZZ}[0]{\mathbb{Z}}
\newcommand{\NN}[0]{\mathbb{N}}
\newcommand{\KK}[0]{\mathbb{K}}
\newcommand{\smallo}[0]{{\scriptstyle \mathcal{O}}}
\DeclarePairedDelimiter\floor{\lfloor}{\rfloor}
\newcommand{\slim}[2]{\lim_{#1\to\infty}{#2}}
\renewcommand{\Re}[0]{\operatorname{Re}}
\renewcommand{\Im}[0]{\operatorname{Im}}
\DeclareMathOperator{\Leb}{Leb}

\setlength{\topmargin}{-.5in} \setlength{\textheight}{9.25in}
\setlength{\oddsidemargin}{0in} \setlength{\textwidth}{6.8in}


\begin{document}

\noindent{\bf Kernfragen - Metrische Räume\hfill Balázs Kossovics \hfill 2022 SS - Analysis II.}


\medskip\hrule
\begin{enumerate}
    %1
    \item Was ist eine Metrik? Gib drei verschiedene Beispiele!

    \textbf{Answer:} Consider $E$ arbitrary set. The function $d: E \times E \to \RR$ is a metric, if satisfies the following conditions:
    \begin{itemize}
        \item $\forall x, y \in E\colon d(x, y) = 0 \Leftrightarrow x = y$
        \item $\forall x, y \in E\colon d(x, y) = d(y, x)$
        \item $\forall x, y, z \in E\colon d(x, z) \le d(x, y) + d(y, z)$
    \end{itemize}

    Examples:
    \begin{itemize}
        \item Consider $E = \RR$ and $d(x, y) = \abs{x -y}$
        \item Consider arbitrary set $E$. The discrete metric is defined as $d(x, y) = \begin{cases}
            0&x=y\\
            1&\text{otherwise}
        \end{cases}$
        \item Consider $\mathcal{BC}([0, 1], \RR)$ and $d(f, g) = \norm{f - g}_{sup}$
    \end{itemize}
    \item Wie sind offene Teilmengen eines metrischen Raumes definiert, wie abgeschlossene Teilmengen?

    \textbf{Answer:} Consider $(E, d)$ metric space, and let $x \in E$. $\forall \varepsilon > 0$ let's define $B_\varepsilon(x) := \left\{ y \in E \bigm | d(x, y) < \varepsilon\right\}$, the open ball of radius $\varepsilon$ centered around $x$. The $A \subset E$ set is open, if $\forall x \in A\colon \exists \varepsilon > 0\colon B_\varepsilon(x) \subset A$. The $B \subset E$ set is closed, if its complement $E \setminus B$ is open.
    \item Sind $\left\{\begin{tabular}{c}
        endliche\\
        abzählbare\\
        beliebige\\
    \end{tabular}\right\}$ $\left\{\begin{tabular}{c}
    Durchschnitte\\
    Vereinigungen
    \end{tabular}\right\}$ $\left\{\begin{tabular}{c}
        offener\\
        abgeschlossener
    \end{tabular}\right\}$ Mengen wieder offen bzw. abgeschlossen?

    \textbf{Answer:} Arbitrary union of open sets is open. Finite intersection of open sets is open. Arbitrary intersection of open sets is in general not open: consider $\RR$ with the standard metric, then $\cap_{n \in \NN} (-\frac{1}{n}, \frac{1}{n}) = \left\{0\right\}$ closed.

    Arbitrary intersection of closed sets is closed. Finite union of closed sets is closed. Arbitrary union of closed sets is in general not closed: consider $\RR$ with the standard metric, then $\cup_{n \in \NN}\left[\frac{1}{n}, 1-\frac{1}{n}\right] = (0, 1)$ open.

    \item Wie sind Abschluss, Inneres und Rand einer Teilmenge eines metrischen Raumes definiert?

    \textbf{Answer:} Consider $(E, d)$ metric space, and let $A \subset E$.

    The closure of $A$ is $\overline{A} = \left\{x \in E \bigm| \forall \varepsilon > 0 \colon B_\varepsilon(x) \cap A \neq \emptyset \right\}$

    The inner of $A$ is $\mathring{A} = \left\{x \in A \bigm| \exists \varepsilon > 0\colon B_\varepsilon(x) \subset A \right\}$

    The boundary of $A$ is $\partial A = \overline{A} \setminus \mathring{A}$
    \item Was sind Abschluss, Inneres und Rand folgender Teilmengen der reellen Zahlen (mit deren Standardmetrik)?


    \hspace*{\fill}
    $\ZZ \hfill \QQ \hfill \cup_{k\in\NN}\left(\frac{1}{k+1}, \frac{1}{k}\right)$
    \hspace*{\fill}

    \textbf{Answer:}
    \begin{itemize}
        \item $\overline{\ZZ} = \ZZ$ because $\ZZ$ only consists of isolated points. $\mathring{\ZZ} = \emptyset$, since $\mathring{\ZZ} = \RR \setminus \overline{\RR \setminus \ZZ} = \RR \setminus \RR = \emptyset$. $\partial \ZZ = \overline{\ZZ} \setminus \mathring{\ZZ} = \ZZ$
        \item $\QQ$ is dense in $\RR$, thus $\overline{\QQ} = \RR$. $\mathring{\QQ} = \RR \setminus \overline{\RR \setminus \QQ} = \RR \setminus \RR = \emptyset$, since the irrationals are also dense in $\RR$. $\partial\QQ = \overline{\QQ} \setminus \mathring{\QQ} = \RR$
        \item $A := \cup_{k\in\NN}\left(\frac{1}{k+1}, \frac{1}{k}\right) = (0, 1) \setminus \left\{\frac{1}{n} \bigm| n \in \NN\right\}$. $\overline{A} = [0, 1]$, $\mathring{A} = A$ since it's an union of open sets, thus it's also open and so the interior is itself. $\partial{A} = \overline{A} \setminus \mathring{A} = \left\{\frac{1}{n} \bigm| n \in \NN\right\} \cup \left\{0\right\}$
    \end{itemize}

    \item Wann ist eine Teilmenge eines metrischen Raumes dicht?

    \textbf{Answer:} Consider $(E, d)$ metric space and $A \subset E$. $A$ is dense in $E$, if $\overline{A} = E$.

    \item Wann ist ein metrischer Raum zusammenhängend?

    \textbf{Answer:} Consider $(E, d)$ metric space. It's connected, if $\forall A, B \subset E\colon A, B$ open, $A \cup B = E, A \cap B = \emptyset \Rightarrow$ either $A = \emptyset$ or $B = \emptyset$
    \item Wie sind Zusammenhangskomponenten eines metrischen Raumes definiert? Wann heißt ein metrischer Raum total unzusammenhängend?

    \textbf{Answer:} Consider $(E, d)$ metric space. For some $x\in E$ we define the connected component of the point $x$ as the union of all $C\subset E$ connected sets, that contain $x$. We denote the connected component of $x$ with $\mathcal{C}(x)$. $(E, d)$ is totally disconnected, if $\forall x\in E\colon \mathcal{C}(x) = \left\{x\right\}$

    \item Sind $\left\{\begin{tabular}{c}
    \text{endliche}\\
    \text{abzählbare}\\
    \text{beliebige}
    \end{tabular}\right\}$ $\left\{\begin{tabular}{c}
    \text{Durchschnitte}\\
    \text{Vereinigungen}
    \end{tabular}\right\}$ zusammenhängender Mengen wieder zusammenhängend? Gib gegebenfalls ein Gegenbeispiel!

    \textbf{Answer:} Union of connected sets is in general not connected: consider $\RR$ with the standard metric, and $[0, 1]$, $[2, 3]$ connected sets. $[0, 1] \cup [2, 3]$ is disconnected. Arbitrary union of connected sets $U_i$ is connected, as long as $\cap_i U_i \neq \emptyset$.

    Intersection of connected sets is not necessarily connected (think in $\RR^2$ about a $C$ and a $I$ shaped open set, intersecting each other only at the ends of the $C$, creating two disconnected sets).

    \item Sind $\left\{\begin{tabular}{c}
        \text{endliche}\\
        \text{abzählbare}\\
        \text{beliebige}
    \end{tabular}\right\}$ $\left\{\begin{tabular}{c}
        \text{Durchschnitte}\\
        \text{Vereinigungen}
    \end{tabular}\right\}$ kompakter Mengen wieder kompakt? Gib gegebenfalls ein Gegenbeispiel!

    \textbf{Answer:} Finite union of compact sets is compact. Arbitrary union of compact sets is in general not compact: $\cup_{n\in\NN}[n, n+1] = \NN$ is not totally bounded, thus also not compact.

    Arbitrary intersection of compact sets is compact.

    \item Warum ist in einem metrischen Raum jede konvergente Folge eine Cauchy-Folge? (Beweise!)

    \textbf{Answer:} Consider $(E, d)$ metric space, and an $(x_n) \in E$ convergent sequence. Let $\lim_{n \to \infty} x_n = x \in E$. Since $x_n$ converges, $\forall \varepsilon > 0\colon \exists N \in \NN\colon \forall n > N\colon d(x_n, x) < \frac{\varepsilon}{2}$. Now $\forall n, m > N\colon d(x_n, x_m) \le d(x_n, x) + d(x, x_m) = d(x_n, x) + d(x_m, x) < \frac{\varepsilon}{2} + \frac{\varepsilon}{2} < \varepsilon$, thus $(x_n)$ is a Cauchy sequence.

    \item Wann heißt ein metrischer Raum vollständig?

    \textbf{Answer:} Consider $(E, d)$ metric space. $E$ is complete, if every Cauchy sequence converges.

    \item Was sind generische Mengen?

    \textbf{Answer:} Consider $(E, d)$ complete metric space and countable many $U_i \subset E~(i \in \mathbb{N})$ open and dense sets. Then $M = \cap_{i \in \NN} U_i$ is dense in $E$. Any $G \supset M$ is called generic, residual or of  Baire second category.

    \item Wie lautet der Satz von Baire?

    \textbf{Answer:} Consider $(E, d)$ complete metric space and countable many $U_i \subset E~(i \in \mathbb{N})$ open and dense sets. Then $\cap_{i \in \NN} U_i$ is dense in $E$.

    \item Wann heißt ein metrischer Raum perfekt?

    \textbf{Answer:} $(E, d)$ is perfect, if it doesn't contain isolated points. Or equivalently: $\forall x \in E\colon x \in \overline{E \setminus \left\{x\right\}}$

    \item Zeige, dass jeder nichtleere, vollständige, perfekte metrische Raum überabzählbar ist.

    \textbf{Answer:} Consider $E$ not empty, complete and perfect metric space. Suppose indirectly, that it's countable: let $(x_n)\colon \NN \to E$ bijection. Let furthermore $A_n = E \setminus \left\{x_n\right\}$. Since $E$ is perfect, every $A_n$ is dense in $E$. Since $\left\{x_n\right\}$ is closed, $A_n$ is open. From Baire's theorem $\cap_{n\in\NN}A_n = \emptyset$ should be dense in $E$, contradiction.

    \item Was versteht man unter einer Cantor-Menge? Gib ein Beispiel an!

    \textbf{Answer:} Let $\emptyset \neq C \subset [0, 1]$. We call $C$ a Cantor set, if $C$ is complete, totally disconnected and perfect.

    \item Wann heißt eine Folge in einem metrischen Raum konvergent? Wann heißt sie Cauchy-Folge?

    \textbf{Answer:} Consider $(E, d)$ metric space, and $(x_n) \in E$ sequence. The $(x_n)$ sequence is convergent, if $\exists x\in E\colon \forall \varepsilon > 0\exists N \in \NN\colon \forall n > N\colon d(x, d_n) < \varepsilon$. We call $x$ the limit of the $(x_n)$ sequence and we say that $(x_n)$ converges agains $x$.

    The $(x_n) \in E$ sequence is a Cauchy sequence, if $\forall \varepsilon > 0\colon \exists N \in \NN\colon \forall m, n > N\colon d(x_n, x_m) < \varepsilon$.

    \item Wann heißt eine Abbildung zwischen metrischen Räumen stetig? Gib vier verschiedene (aber natürlich äquivalente) Definitionen!

    \textbf{Answer:} Consider $(E_1, d_1)$ and $(E_2, d_2)$ metric spaces. The $f\colon E_1 \to E_2$ function is continuous in $a \in E_1$ if $\forall \varepsilon > 0\colon \exists \delta > 0\colon \forall x \in E_1\colon d_1(a, x) < \delta\colon d_2(f(a), f(x)) < \varepsilon$.

    The $f\colon E_1 \to E_2$ is continuous if and only if one of the following equivalent characterisation holds (and if one holds, then all the other hold as well):
    \begin{enumerate}
        \item $f$ is continuous in $\forall a \in E_1$
        \item $\forall A \subset E_1\colon f(\overline{A}) \subset \overline{f(A)}$
        \item $\forall A \subset E_2$ closed: $f^{-1}(A)$ is closed
        \item $\forall A \subset E_2$ open: $f^{-1}(A)$ is open
    \end{enumerate}

    \item Sind unter einer stetigen Abbildung $f: E \to E^\prime$ zwischen metrischen Räumen die $\left\{\begin{tabular}{c}
        \text{Bilder}\\
        \text{Urbilder}
    \end{tabular}\right\}$ $\left\{\begin{tabular}{c}
    \text{offener}\\
    \text{abgeschlossener}\\
    \text{vollstängiger}\\
    \text{zusammenhängender}\\
    \text{kompakter}
    \end{tabular}\right\}$ Teilmengen wieder $\left\{\begin{tabular}{c}
        \text{offener}\\
        \text{abgeschlossener}\\
        \text{vollstäntiger}\\
        \text{zusammenhängender}\\
        \text{kompakter}
        \end{tabular}\right\}$?
    Gib gegebenfalls ein Gegenbeispiel!

    \textbf{Answer:}
    \begin{itemize}
        \item Image of an open subset is not necessarily open. Let $E = \RR$ with the standard metric, $f\colon \RR \to \RR, x \mapsto 1$ the constant one function and consider $A \subset E$ open. $f(A) = \left\{1\right\}$ is closed.
        \item Image of a closed subset is not necessarily closed. Consider $f\colon \RR \to \RR, x \mapsto \frac{1}{1+e^{-x}}$. The subset $\RR \subseteq \RR$ is closed, but $f(\RR) = (0, 1)$ is open.
        \item Image of a complete subset is not necessarily complete. See previous example.
        \item Image of connected or compact sets are again connected or compact, respectively.
        \item Preimage of an open set is open.
        \item Preimage of a closed set is closed.
        \item Preimage of a complete set is not necessarily complete. Consider $(0, 1)$ metric space, let $f\colon (0, 1) \to \RR$ be the constant $1$ function. The preimage of the complete set $\{1\}$ is $(0, 1)$, which is not complete.
        \item Preimage of connected sets is not necessarily         \item Preimage of connected sets is not necessarily complete. Consider the $(0, 1) \cup (2, 3)$ metric space, and the constant $1$ function defined on this space. The preimage of the connected $\{1\}$ set is disconnected.
        . Consider the $(0, 1) \cup (2, 3)$ metric space, and the constant $1$ function defined on this space. The preimage of the connected $\{1\}$ set is disconnected.
        \item Preimage of compact sets is not necessarily compact. Consider the $\RR$ metric space, and the constant $1$ function. The preimage of the compact $\{1\}$ set is not bounded, thus also not compact.
    \end{itemize}
    \item Wann heißen zwei Metriken äquivalent?

    \textbf{Answer:} Consider $(E_1, d_1)$ and $(E_2, d_2)$ metric spaces. The $f\colon E_1 \to E_2$ function is a homeomorphism, if $f$ is bijective, and both $f$ and $f^{-1}$ are continuous. We say, that $E_1$ and $E_2$ metric spaces are equivalent, if the $id: E_1 \to E_2, x \mapsto x$ function is a homeomorphism.

    \item Wann heißen zwei Normen äquivalent?

    \textbf{Answer:} Consider $V$ vectorspace with two different $\norm{.}_1$ and $\norm{.}_2$. We say, that they are equivalent, if $\exists c, C \in \RR^+$, such that $\forall v \in V\colon c \norm{v}_2 < \norm{v}_1 < C \norm{v}_2$

    \item Sei $E$ ein metrischer Raum. Formuliere und beweise den Zwischenwertsatz für stetige Abbildungen $f\colon E \to \RR$

    \textbf{Answer:} Let $E$ connected, $f\colon E \to \RR$ continuous and $a, b \in E$. Then $f$ takes on every value between $f(a)$ and $f(b)$. \textit{Proof:} $E$ is connected, thus so is $f(E)$. In $\RR$ connected sets are exactly the intervals, thus it'll take on any value between $f(a)$ and $f(b)$.

    \item Gib drei verschiedene (aber natürlich äquivalente) Definitionen von kompakten metrischen Räumen! Gib außerdem je ein Beispiel eines kompakten und eines nicht kompakten Raumes!

    \textbf{Answer:} Consider $(E, d)$ metric space. The following statements are equivalent characterisation of the compactness of $E$:
    \begin{itemize}
        \item Every open cover of $E$ has a finite subcover.
        \item Every sequence in $E$ has a convergent subsequence.
        \item $E$ is complete and totally bounded.
    \end{itemize}

    The $[0, 1]$ space with the standard metric of $\RR$ is compact, but the $(0, 1)$ space with the same metric is not.

    \item Sei E ein metrischer Raum. Wann heißt eine Teilmenge $A \subset E$ relativ kompakt?

    \textbf{Answer:} $A$ is relative compact, if $\overline{A}$ is compact.

    \item Wie lassen sich die kompakten Teilmengen des $\RR^n$
    charakterisieren?

    \textbf{Answer:} In $\RR^n$ the subset $K \subset \RR^n$ is bounded if and only if it's totally bounded.Thus closed and bounded subsets $K \subset \RR^n$ are compact.

    \item Wie lautet der Satz von Arzela-Ascoli?

    \textbf{Answer:} Consider $(E_1, d_1), (E_2, d_2)$ metric spaces and $\mathcal{F} := \left\{f \colon E_1 \to E_2 \bigm| f\text{ continuous}\right\}$. We say, that the set $\mathcal{F}$ is equicontinuous, if $\forall a\in E_1\colon \forall \varepsilon > 0\colon \exists \delta > 0\colon \forall a\in E_1\colon \abs{x - a} < \delta\colon \abs{f(x) - f(a)} < \varepsilon~(\forall f \in \mathcal{F})$. That is: $\delta$ only depends on $a \in E_1$ and $\epsilon > 0$, but not on $f \in \mathcal{F}$.

    Consider now $(E, d)$ compact metric space, and $(F, \norm{.})$ Banach space. The $\mathcal{F} \subset \mathcal{BC}^0(E, F)$ set is relative compact, if and only if both of the following statement hold:
    \begin{itemize}
        \item $\mathcal{F}$ is equicontinuous
        \item $\forall x \in E: \left\{f(x) \bigm| f \in \mathcal{F}\right\}$ is relative compact
    \end{itemize}

    \item Wann sagt man, dass eine Abbildung Lipschitz-stetig ist? Wann ist eine Lipschitzstetige Abbildung eine Kontraktion?

    \textbf{Answer:} Consider $(E_1, d_1)$ and $(E_2, d_2)$ metric spaces. $f\colon E_1 \to E_2$ is Lipschitz continuous, if $\exists L > 0\colon \forall x, y\in E_1\colon d_2(f(x), f(y)) < L d_2(x, y)$. If $L < 1$, then we call $f$ a contraction.

    \item Wie lautet der Banachsche Fixpunktsatz?

    \textbf{Answer:} Consider $(E, d)$ complete metric space and $f\colon E \to E$ contraction with Lipschitz constant $L < 1$. Then
    \begin{itemize}
        \item $f$ has exactly one fixed point $x_\star \in E\colon f(x_\star) = x_\star$
        \item $\forall x_0 \in E$ converges the $x_{n+1} = f(x_n)~(\forall n \in \NN)$ against $x_\star$
        \item the following error estimates hold:
        \begin{itemize}
            \item $d(x_n, x_\star) \le L^n d(x_0, x_\star)$
            \item $d(x_n, x_\star) \le \frac{L^n}{1-L}d(x_0, x_1)$
        \end{itemize}
    \end{itemize}
    \item Sei $E$ ein metrischer Raum und $f_i \colon E \to E$ stetig. Wann sagt man, dass $\emptyset \neq A \subset E$ selbstähnlich ist? Was ist ein \textit{iterated function system}?

    \textbf{Answer:} Consider $f_i \colon E \to E$ stetig ($i = 1,\dots,m \ge 2$). $A$ is self similar, if $A = \cup_{i=1}^m f_i(A)$. If $E$ is a complete metric space and $f_i$'s are contractions ($\forall i = 1,\dots,m \ge 2$), then $\exists ! A \subset E$ compact, such that $A = \cup_{i=1}^m f_i(A)$ and we call such a set of functions and iterated function system.
\end{enumerate}

\end{document}
