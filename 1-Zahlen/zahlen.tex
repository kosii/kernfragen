\documentclass[11pt]{article}

\usepackage{amsmath,amssymb,amsfonts}
\usepackage{graphicx}
\usepackage{mathtools}
\usepackage[dvipsnames]{xcolor}
\usepackage{}


\newcommand{\norm}[1]{\left\lVert#1\right\rVert}
\newcommand{\abs}[1]{\left|#1\right|}
\newcommand{\sumn}[4]{\sum_{#1=#2}^{#3}{#4}}
\newcommand{\RR}[0]{\mathbb{R}}
\newcommand{\CC}[0]{\mathbb{C}}
\newcommand{\QQ}[0]{\mathbb{Q}}
\newcommand{\ZZ}[0]{\mathbb{Z}}
\newcommand{\NN}[0]{\mathbb{N}}
\newcommand{\KK}[0]{\mathbb{K}}
\newcommand{\smallo}[0]{{\scriptstyle \mathcal{O}}}
\DeclarePairedDelimiter\floor{\lfloor}{\rfloor}
\newcommand{\slim}[2]{\lim_{#1\to\infty}{#2}}
\renewcommand{\Re}[0]{\operatorname{Re}}
\renewcommand{\Im}[0]{\operatorname{Im}}

\setlength{\topmargin}{-.5in} \setlength{\textheight}{9.25in}
\setlength{\oddsidemargin}{0in} \setlength{\textwidth}{6.8in}


\begin{document}

\noindent{\bf Kernfragen - Zahlen\hfill Balázs Kossovics \hfill WS21-22 Analysis I.}


\medskip\hrule
\begin{enumerate}
    \item Wie lautet das \textit{Wohlordnungsprinzip}?

    \textbf{Answer:} Any nonempty subset of the natural numbers has a unique smallest element, that is: $\forall M \subset \mathbb{N}~(\emptyset \neq M) \exists! m \in M\colon \forall n \in M\colon m \le n$


    \item Was ist \textit{Vollständige Induktion}?

    \textbf{Answer:} Consider a subset $A \subset \mathbb{N}$. If $1 \in A$ and $(n \in A) \Rightarrow (n+1) \in A\phantom{0}(\forall n \in \mathbb{N}, n > 0)$, then $A = \mathbb{N}$. This is the same theorem as in the lecture, applied to the truth set of a prediate.

    \item Zeige mittels Vollständiger Induktion, dass für alle natürlichen Zahlen $n$ gilt:
    $$\sumn{k}{1}{n}{k} = \frac{n(n+1)}{2}$$

    \textbf{Answer:}
    \begin{itemize}
        \item $n = 1$:

        $\sum_{k=1}^k{k} = 1 = \frac{1(1+1)}{2}$
        \item $n \Rightarrow n + 1$:

        $\sum_{k=1}^{n+1}{k} = \sum_{k=1}^n{k} + (n+1) = \frac{n(n+1)}{2} + (n+1) = \frac{(n+1)(n + 2)}{2}$
    \end{itemize}

    % Question 4
    \item Was ist die \textit{Zifferndarstellung} einer natürlichen Zahl $n$ zur Basis $b$?

    \textbf{Answer:} Let $n \in \mathbb{N}$ and $b \ge 2$. Then there are some unique numbers $k \in \mathbb{N}$ and $a_0, a_1, \dots, a_k \in \left\{0, \dots, b-1\right\}$ with $a_k \neq 0$   such that $\sum_{i=0}^k {a_i b^i} = n$


    % Question 5
    \item Seien $A$ und $B$ Mengen. Wann nennt man eine Funktion $f\colon A \to B$ \textit{injektiv}, wann
    \textit{surjektiv}, wann \textit{bijektiv}?

    \textbf{Answer:}
    \begin{itemize}
        \item \textit{injective}: $\forall a, b \in A\colon f(a) = f(b) \Rightarrow a = b$
        \item \textit{surjective}: $\forall b\in B\colon \exists a\in A\colon f(a) = b$
        \item \text{bijective}: whenever $f$ is injective and surjective
    \end{itemize}

    \item Was sind die \textit{Binomialkoeffizienten} $\binom{n}{k}$, und welche Rekursionsformel erfüllen sie? Wie lässt sich die Rekursionsformel \textit{kombinatorisch} (d.h. als Abzählung von Teilmengen) interpretieren?

    \textbf{Answer:}
    The binomial coefficients of $n$ are the coefficients that occur when raising two numbers $x, y \in \mathbb{R}$ to the $n^\text{th}\phantom{0}(n\in \mathbb{N})$ power:
    $$(x+y)^n = \sum_{k=0}^n {\binom{n}{k}x^k y^{n-k}}$$

    Combinatorically $\binom{n}{k}$ is the number of different subsets of size $k$ of a set of size $n$.

    For $n \ge 0$, $0 \le k \le n$ holds $\binom{n+1}{k} = \binom{n}{k} + \binom{n}{k-1}$, where $\binom{n}{0} = 1$ and $\binom{n}{n} = 1$. The binomial coefficient furthermore fulfills the following: $\binom{n}{k} = \frac{n!}{(n-k)! k!}$


    \item Wie lautet der \textit{Binomische Lehrsatz}? Wie folgt daraus, dass
    $$\sumn{k}{0}{n}{\binom{n}{k}} = 2^n\text{ ?}$$

    \textbf{Answer:} $\forall x, y \in \mathbb{R}, n\in\mathbb{N}\colon (x+y)^n = \sum_{k=0}^n {\binom{n}{k}x^k y^{n-k}}$ where $x^0 = 1, 0^0 = 1, 0! = 1$
    
    With $x = y = 1\colon 2^n = \sum_{k=0}^n {\binom{n}{k}1^k 1^{n-k}} = \sum_{k=0}^n {\binom{n}{k}}$
    

    \item Was ist eine \textit{rekursiv} definierte Folge? Gib Beispiele an.

    \textbf{Answer:} A sequence $(x_n) \in \mathbb{R}$ is defined recursively if there is a $k \in \mathbb{N}$, a function $f\colon \mathbb{R}^k \to \mathbb{R}$ and initial elements $x_1, x_2, \dots, x_k$ such that $x_{n+1} = f(x_n, x_{n-1}, \dots, x_{n - k + 1})\phantom{0}(\forall n > k)$. A recursively defined sequence is well defined. This definition is equivalent to the one given in the lecture with $k=1$.
    
    With $k = 2$, $x_1 = 1, x_2 = 2$ and $f\colon \RR^2 \to \RR$ with $f(a, b) = a + b$ consider the $x_{n+1} = f(x_n, x_{n-1})$ sequence, known as the Fibonacci-sequence.

    \item Was sind \textit{endliche}, \textit{abzählbare} bzw. \textit{überabzählbare} Mengen? Gib Beispiele an.

    \textbf{Answer:} A set $B$ has a cardinality of $n \in \NN$ when there is a bijection between $B$ and $\{1, 2, \dots, n\}$. We call a set $B$
    \begin{enumerate}
        \item \label{cardinality:finite} finite, whenever $\left(B\right) = n \in \NN$. Example: $B = \{42\}$
        \item \label{cardinality:countable} countable, whenever there is a bijection $\phi: B \to \NN$. Example: $B = \QQ$
        \item uncountably infinite, whenever neither \ref{cardinality:finite} or \ref{cardinality:countable} holds. Example: $B = [0, 1]$
    \end{enumerate}

    \item Zeige, dass $\NN \times \NN$ abzählbar ist.

    \textbf{Answer:} Consider the same bijection as in Question 7. of Sequences (Lecture 5 for details).

    % Question 11
    \item Sei $A \subseteq \RR$. Was ist eine \textit{obere Schranke} für $A$? Wann heißt $A$ nach oben beschränkt?

    \textbf{Answer:} $a \in \RR$ is an upper bound of $A$ if $\forall x\in A\colon x \le a$. $A$ is bounded above if it has an upper bound. Notation: $A \le a$

    \item Sei $A \subseteq \RR$. Wie sind \textit{Supremum} und \textit{Infimum} von $A$ definiert? Wann besitzt $A$ ein Supremum, wann ein Maximum?

    \textbf{Answer:} Supremum is the lowest upper bound. $\alpha = \sup A$ if $\alpha$ is an upper bound and for any other $a$ upper bound of $\alpha \le a$. Every $A$ set that is bounded above has a lowest upper bound. $\exists \max A \Leftrightarrow \sup A \in A$ and whenever $\max A$ exists $\max A = \sup A$.

    \item Sei $B = \left\{1, \frac{1}{2}, \frac{1}{3}, \frac{1}{4}, \dots\right\}$. Bestimme $\inf B$ und $\sup B$.

    \textbf{Answer:} $\sup B = 1, \inf B = 0$

    \item Gib ein Beispiel einer Menge reeller Zahlen an, die ein Supremum aber kein Maximum besitzt.

    \textbf{Answer:} $[0, 1) \subset \RR$

    \item Was ist ein \textit{Dedekindscher Schnitt}?

    \textbf{Answer:} Consider $\emptyset \neq L, R \subset \RR$. $(L|R)$ is called a Dedekind-cut in $\RR$ whenever the following conditions are both satisfied:
    \begin{itemize}
        \item $L < R$
        \item $L \cup R = \RR$
    \end{itemize}

    A $t\in\RR$ number is called \textit{Trennzahl} of $(L|R)$ if $L \le t \le R$ holds.

    The above is also defined in $\QQ$.

    \item Wie lautet das \textit{Vollständigkeitsaxiom} der reellen Zahlen?

    \textbf{Answer:} Every $(L|R)$ Dedekind-cut in $R$ has one and only one Trennzahl (in $\RR$). In comparison not every Dedekind-cut in $\QQ$ has a Trennzahl in $\QQ$.

    \item Definiere die \textit{komplexen Zahlen} als Paare reeller Zahlen mit geeigneten Additions- und Multiplikationsregeln.

    \textbf{Answer:} For $\CC = \RR^2$ we obtain a (commutative) field with the following addition and multiplication operations:
    \begin{itemize}
        \item $\forall (x_1, y_1), (x_2, y_2) \in \CC\colon (x_1, y_1) + (x_2, y_2) = (x_1 + x_2, y_1 + y_2)$
        \item $\forall (x_1, y_1), (x_2, y_2) \in \CC\colon (x_1, y_1) \cdot (x_2, y_2) = (x_1\cdot x_2 - y_1\cdot y_2, x_1 \cdot y_2 + y_1 \cdot x_2)$
    \end{itemize}

    \item Was ist der \textit{Betrag} einer komplexen Zahl $z \in \CC$?

    \textbf{Answer:} $\forall z \in \CC, z = a + ib~(a, b\in \RR)\colon \abs{z} = \sqrt{a^2 + b^2} = \sqrt{\Re(z)^2 + \Im(z)^2}$

    \item Was ist die zu $z$ \textit{komplex konjugierte} Zahl $\overline{z}$ ?

    \textbf{Answer:} $\forall z \in \CC\colon \overline{z} = \Re(z) - i\Im(z)$

    \item Wie hängen $z$ und $\overline{z}$ mit $\abs{z}$ zusammen?

    \textbf{Answer:} $\forall z \in \CC\colon \abs{z}^2 = z \overline{z}$

    \item Wie lautet der \textit{Fundamentalsatz der Algebra}?

    \textbf{Answer:} Every polynomial of degree $n$ in one variable with complex coefficients has exactly $n$ (counted with multiplicities) complex roots. If $p(x) = \sumn{k}{0}{n}{a_k x^k}~(a_n \neq 0)$ and the roots are $z_1, z_2, \dots, z_n \in \CC$ then we get through polynomial division $p(x) = a_n \prod_{k=0}^n(z - z_k)~(\forall z\in \CC)$

    \item Wie hängen komplexe Zahlen $z = x + iy \in \CC$ mit Drehstreckungen in $\RR$ zusammen?

    \textbf{Answer:} Consider $C = \left\{\left(\begin{array}{cc}x&-y\\y&x\end{array}\right)\colon x,y \in \RR \right\}$, the set of rotation-dilation matrices, with the standard matrix operations (addition, multiplication). Then with $\phi\colon\CC \to C$ bijection where $\phi(z)~=~\left(\begin{array}{cc}\Re(z)&-\Im(z)\\\Im(z)&\Re(x)\end{array}\right)$ we get an isomorphism between $C$ and $\CC$.

    \item Was sind Quaternionen?

    \textbf{Answer:} Consider $Q = \left\{\left(\begin{array}{cc}a&-\overline{b}\\b&\overline{a}\end{array}\right)\colon a, b \in \CC \right\}$, the set of quaterions. Through the standard matrix operations on $Q$ we obtain a non-commutative field.


\end{enumerate}

\end{document}
